\csname @openrightfalse\endcsname
\chapter{Piecewise linear geometry}


A \index{polyhedral space}\emph{polyhedral space} is complete length-metric space which admits a locally finite triangulation 
such that each simplex is isometric to a simplex in a Euclidean space.
By {}\emph{triangulation} of polyhedral space we always understand triangulation as above. 

A point in a polyhedral space is called \index{regular point}\emph{regular} if it has a neighborhood isometric to an open set in a Euclidean space;
otherwise it called {}\emph{singular}.

If above we exchange the Euclidean spaces to the unit spheres or the hyperbolic spaces,
we arrive to the definition of {}\emph{spherical} and correspondingly {}\emph{hyperbolic polyhedral spaces}.

The term \index{piecewise}\emph{piecewise} typically mean that there is a triangulation with some property on each triangle.
For example,  if $P$ and $Q$ are polyhedral spaces, then
\begin{itemize}
\item a map $f\:P\to Q$ is called {}\emph{piecewise distance preserving} if there is a triangulation $\mathcal{T}$ of $P$ such that at any simplex $\Delta\in \mathcal{T}$ the restriction $f|_\Delta$ is distance preserving,
\item a map $h\:P\z\to Q$  is called \emph{piecewise linear} if both spaces $P$ and $Q$ admit triangulations such that each simplex of $P$ is mapped to a simplex of $Q$ by an affine map.
In particular, a {}\emph{piecewise linear homeomorphism} is a piecewise linear map which is a homeomorphism.\label{piecewise linear map}
\end{itemize}





%%%%%%%%%%%%%%%%%%%%%%%%%%%%%%%%%%%%%%%%%
\subsection*{Spherical arm lemma}

\begin{pr}{}{Spherical arm lemma}\label{Spherical arm lemma}
Let $A=a_1a_2\dots a_n$ and $B=b_1b_2\dots b_n$ be two simple spherical polygons 
with equal corresponding sides.
Assume $A$ lies in a hemisphere and $\measuredangle a_i\ge\measuredangle b_i$ for each $i$.
Show that $A$ is congruent to $B$.
\end{pr}

%%%%%%%%%%%%%%%%%%%%%%%%%%%%%%%%%%%%%%%%%%%%%%%%%%
\parit{Semisolution.}
Let us cut the polygon $A$ from the sphere and glue instead the polygon $B$.
Denote by $\Sigma$ the obtained spherical polyhedral space.
Note that 
\begin{itemize}
\item $\Sigma$ is homeomorphic $\mathbb S^2$.
\item $\Sigma$ has curvature $\ge 1$ in the sense of Alexandrov; that is, the total angle around each singular point is less than $2\cdot \pi$.
\item All the singular points of $\Sigma$ 
lie outside of an isometric copy of a hemisphere $\mathbb{S}^2_+\subset \Sigma$
\end{itemize}

Denote by $n$ the number of singular points in $\Sigma$.
It is sufficient to show that $n=0$.

Assume the contrary; that is $n\ge 1$.
We will arrive to a contradiction applying induction on $n$.
The base case $n=1$ is trivial; 
that is, $\Sigma$ cannot have single singular point.

Now assume $\Sigma$ has $n>1$ singular points.
Choose two singular points $p, q$,
cut $\Sigma$ along a geodesic $[pq]$.
Show that the hole hole can be patched so that we obtain a new polyhedral space $\Sigma'$ of the same type but wit $n-1$ singular points.
(The needed patch is obtained by doubling a
spherical triangle along two sides.)

By induction hypothesis $\Sigma'$ does not exist. Hence the result follow.
\qeds

The problem is due to Victor Zalgaller \cite[see][]{zalgaller-shperical-polygon};
the result of Victor Toponogov in \cite{toponogov} gives a smooth analog of this statement.
The patch construction above was introduced by 
Aleksandr Alexandrov
in his proof of convex embeddability of polyhedrons
\cite[see][VI, \S7]{alexandrov1948}.

Here is an alternative end of proof from \cite{panov-petrunin}:
By Alexandrov embedding theorem, $\Sigma$ is isometric to the surface of convex polyhedron $P$ in the unit 3-dimensional sphere $\mathbb S^3$. 
The center of hemisphere has to lie in a facet, say $F$ of $P$.
It remains to note that $F$ contains the equator and therefore $P$ has to be hemisphere in $\mathbb S^3$ or intersection of two hemispheres.
In both cases its surface is isometric to $\mathbb S^2$.


%%%%%%%%%%%%%%%%%%%%%%%%%%%%%%%%%%%%%%%%%
\subsection*{Triangulation of 3-sphere}

\begin{pr}{}{Triangulation of 3-sphere}\label{4-poly}
Construct a triangulation of $\mathbb{S}^3$ 
such with $100$ vertices
such that any two vertices are connected by an edge.
\end{pr}

%%%%%%%%%%%%%%%%%%%%%%%%%%%%%%%%%%%%%%%%%
\subsection*{Folding problem}

\begin{pr}{}{Folding problem} \label{Folding problem}
Let $P$ be a compact $2$-dimensional 
polyhedral space. 
Construct a 
piecewise distance preserving map
$f\:P\to \RR^2$.
\end{pr}

%%%%%%%%%%%%%%%%%%%%%%%%%%%%%%%%%%%%%%%%%
\subsection*{Piecewise linear extension}

\begin{pr}{}{Piecewise linear extension} \label{iso-kirzhbraun}
Prove that any 1-Lipschitz map from a finite subset $F\subset \RR^2$
to 
$\RR^2$ can be extended to a 
piecewise distance preserving map
$\RR^2\to\RR^2$.
\end{pr}

%%%%%%%%%%%%%%%%%%%%%%%%%%%%%%%%%%%%%%%%%
\subsection*{Closed polyhedral surface}

\begin{pr}{}{Closed polyhedral surface}\label{Closed polyhedral surface}
Construct a closed polyhedral surface in $\RR^3$ with nonpositive curvature;
that is, the total angle around each vertex is at least $2\cdot\pi$.
\end{pr}

%%%%%%%%%%%%%%%%%%%%%%%%%%%%%%%%%%%%%%%%%
\subsection*{Minimal polyhedral disc}

By a polyhedral disc in $\RR^3$
we understand a triangulation of a plane polygon with a map in $\RR^3$ which is affine on each triangle.
The area of the polyhedral disc is defined as the sum of areas of the images of the triangles in the triangulation.

\begin{pr}{}{Minimal polyhedral disc}\label{Minimal polyhedral disc}
Consider the  class of polyhedral discs glued from $n$ triangles in $\RR^3$ 
with fixed broken line as the boundary.
Let $\Sigma_n$ be a disc of minimal area in this class.
Show that $\Sigma_n$ is  \index{saddle surface}\emph{saddle};
that is, a plane can not cut all the edges comming from one of the interior vertices of $\Sigma_n$.
\end{pr}

%%%%%%%%%%%%%%%%%%%%%%%%%%%%%%%%%%%%%%%%%
\subsection*{Coherent triangulation\easy}

A triangulation of a convex polygon is called coherent if there is a convex function which is linear on each triangle and changes the gradient on every edge of the triangulation.

\begin{pr}{\easy}{Coherent triangulation}\label{Coherent triangulation} 
Find a non-coherent triangulation of a triangle.
\end{pr}

%%%%%%%%%%%%%%%%%%%%%%%%%%%%%%%%%%%%%%%%%
\subsection*{A sphere with one edge\hard}

\begin{pr}{\hard}{A sphere with one edge}\label{panov-S^3} 
Given  a spherical polyhedral space $P$,
denote by $P_s$ the subset of its 
singular points.

Construct spherical polyhedral space $P$ which is homeomorphic to $\mathbb{S}^3$ and such that $P_s$ is formed by a knotted circle.
Show that in such an example the total length of $P_s$ can be arbitrary large and the angle around $P_s$ can be made strictly less than $2\cdot\pi$.
\end{pr}

%%%%%%%%%%%%%%%%%%%%%%%%%%%%%%%%%%%%%%%%%
\subsection*{Triangulation of a torus}

\begin{pr}{}{Triangulation of a torus}\label{Triangulation of a torus}
Show that the torus does not admit a triangulation 
such that one vertex has 5 edges,
one has 7 edges and 
all other vertexes have 
6 edges. 
\end{pr}


%%%%%%%%%%%%%%%%%%%%%%%%%%%%%%%%%%%%%%%%%
\subsection*{No simple geodesics\easy}

\begin{pr}{\easy}{No simple geodesics}\label{No simple geodesics}
Construct a convex polyhedron $P$ whose surface 
does not have a closed simple geodesic.
\end{pr}

\section*{Semisolutions}
%%%%%%%%%%%%%%%%%%%%%%%%%%%%%%%%%%%%%%%%%%%%%%%%%%


%%%%%%%%%%%%%%%%%%%%%%%%%%%%%%%%%%%%%%%%%%%%%%%%%%
\parbf{Triangulation of 3-sphere.}
Choose 100 distinct points $x_1,x_2,\z\dots,x_{100}$
on the {}\emph{moment curve} 
\[\gamma\:t\mapsto (t,t^2,t^3,t^4)\] 
in $\RR^4$.
Let $P$ be the convex hull of $\{x_1,x_2,\z\dots,x_{100}\}$.

Prove that for any two points $x_i$ and $x_j$ there is a hyperplane $H$ in $\RR^4$ which pass through $x_i$ and $x_i$ and leaves $\gamma$ on one side.
The latter statement implies that any two vertices $x_i$ and $x_j$
of $P$ are connected by an edge.

The statement follows
since the surface of $P$ is homeomorphic to $\mathbb{S}^2$.
\qeds

The polyhedron $P$ above is an example 
of so called \index{cyclic polytope}\emph{cyclic polytopes}.

%%%%%%%%%%%%%%%%%%%%%%%%%%%%%%%%%%%%%%%%%%%%%%%%%%
\parbf{Folding problem.}
Given a triangulation of $P$
consider the Voronoi domain $V_v$ for each vertex $v$.
Prove that the triangulation can be subdivided if necessary
so that Voronoi domain of each vertex is isometric to a convex subset in the cone with vertex corresponding to the tip.

Note that the boundaries of all the Voronoi domains form a graph with straight edges.
One can triangulate $P$ so that each triangle has such edge as the base 
and the opposite vertex is the center of an adjusted Voronoi domain; 
such a vertex will be called {}\emph{main} vertex of the triangle.


\begin{wrapfigure}[8]{o}{43 mm}
\begin{lpic}[t(-0 mm),b(0 mm),r(0 mm),l(0 mm)]{pics/zalgaller(1)}
\lbl[tl]{12.5,13.5;$v$}
\lbl[tr]{36.5,24;$w$}
\lbl[l]{33.5,2;$a$}
\lbl[b]{21,34;$b$}
\lbl[tl]{22,17;$x$}
\lbl{26,11;$\triangle$}
\lbl[w]{11,8;$V_v$}
\lbl[w]{36,30;$V_w$}
\lbl[tl]{17,17;$\rho$}
\lbl[bl]{17,19;{\small $\theta$}}
\end{lpic}
\end{wrapfigure}

Fix a solid triangle $\triangle=[vab]$ in the constructed triangulation; 
let $v$ be its main vertex.
Given a point 
$x\in  \triangle$, set 
\begin{align*}
\rho(x)&=|x-v|
\intertext{and}
\theta(x)&=\min \{\measuredangle \hinge vax,\measuredangle\hinge vbx\}.
\end{align*}
Map $x$ to the plane the point with polar coordinates $(\rho(x),\theta(x))$.

Note that for each triangle $\triangle$, 
the constructed map $\triangle\to\RR^2$ is piecewise distance preserving.
It remains to check that these maps agree on the common sides of the triangles.
\qeds


This construction was given by Victor Zalgaller in \cite{zalgaller-polyhedra}, 
see also \cite{petrunin-yashinsky}.
Svetlana Krat generalized the statement to the higher dimensions \cite[see][]{krat}.



%%%%%%%%%%%%%%%%%%%%%%%%%%%%%%%%%%%%%%%%%%%%%%%%%%
\parbf{Piecewise linear extension.}
Let $a_1,a_2,\dots,a_n$
and $b_1$, $b_2,\z\dots,b_n$
be two collections of points in $\RR^2$
such that $|a_i-a_j|\ge |b_i-b_j|$ for all pairs $i$, $j$.
We need to construct a piecewise distance preserving map $f\:\RR^2\to\RR^2$
such that $f(a_i)=b_i$ for each $i$.

Assume that the problem is already solved if $n<m$;
let us do the case $n=m$.
By assumption, 
there is a piecewise liner length-preserving map $f\:\RR^2\to\RR^2$
such that $f(a_i)=b_i$ for each $i>1$.
Consider the set 
\[\Omega=\set{x\in\RR^2}{|f(x)-b_1|>|x-a_1|}.\]
If $\Omega=\emptyset$,
then $f(a_1)=b_1$; 
that is, $f$ is a solution.

Assume $\Omega\ne\emptyset$. 
Prove that $\Omega$ is the interior of a polygon
which is star-shaped with respect to $a_1$.
Redefine the map $f$ inside $\Omega$ so that it remains piecewise distance preserving and $f(a_1)=b_1$.
\qeds

The same proof works in all dimensions.

The statement was proved by Ulrich Brehm in \cite{brehm}
and rediscovered by Arseniy Akopyan and Alexey Tarasov in \cite{akopyan-tarasov},
see also \cite{petrunin-yashinsky}.
The idea in the proof is the same as in the proof of Kirszbraun's theorem given in \cite{valentine}.

%%%%%%%%%%%%%%%%%%%%%%%%%%%%%%%%%%%%%%%%%%%%%%%%%%
\parbf{Closed polyhedral surface.}
Start with you favorite convex polyhedron $K$.
Assume that the interior of $K$ contains the origin $0\in\RR^3$.
Remove from $K$ the interior of $K'=\tfrac56\cdot K$.

\begin{wrapfigure}{o}{37 mm}
\begin{lpic}[t(-0
 mm),b(-0 mm),r(0 mm),l(0 mm)]{pics/octahedron-with-holes(1)}
\end{lpic}
\end{wrapfigure}

Note that one can drill from each vertex of $K$ a polyhedral tunnel to the corresponding vertex $K'$
so that the surface of obtained nonconvex polytope is a solution.

(On the diagram you see the result of this construction for the octahedron.)
\qeds


The construction above produce a surface of genus at least 3.
One can also construct a polyhedral surface in $\RR^3$
which is isometric to a flat torus.
It follows from very general result of Burgo and Zalgaller \cite[see][]{burago-zalgaller:pl}.
They show in particular that any 1-Lipschitz smooth embedding of flat torus in $\RR^3$ can be approximated by piecewise linear isometric embedding.

\begin{wrapfigure}[4]{r}{20 mm}
\begin{lpic}[t(-7 mm),b(0 mm),r(0 mm),l(0 mm)]{pics/cylinder(1)}
\end{lpic}
\end{wrapfigure}

The following construction is more direct;
it is a bent version of so called \index{Schwarz boot}\emph{Schwarz boot} \cite[see][]{schwarz1890definition}.
Construct an isometric piecewise linear embedding 
of cylinder as shown on the diagram
such that the planes thru the boundary triangles meet at the angle $\tfrac\pi n$ for a positive integer $n$.
It remains to reflect the obtained surface several times in the planes through the boundary triangles.

The problem suggested by Jaros{\l}aw K\k{e}dra.

%%%%%%%%%%%%%%%%%%%%%%%%%%%%%%%%%%%%%%%%%%%%%%%%%%
\parbf{Minimal polyhedral disc.}
Arguing by contradiction, 
assume a polyhedral disc $\Sigma_n$ minimize the area but not saddle.

Prove that 
one can move one of the vertices of $\Sigma_n$ in such a way that the lengths of all edges starting at this vertex decrease.

Prove that if, 
by this deformation, 
the area does not decease,
then there are two adjusted triangles in the triangulation, 
say $[pxy]$ and $[qxy]$
such that 
\[\measuredangle \hinge pxy+\measuredangle \hinge qxy> \pi.\]


Finally show that in this case exchanging triangles $[pxy]$ and $[qxy]$
to the triangles $[xpq]$ and $[ypq]$
leads to a polyhedral surface with smaller area.
That is, $\Sigma_n$ is not area minimizing, a contradiction.
\qeds

This problem is discussed in \cite{petrunin-monthly}.

For general polyhedral surface, the deformation which decrease the lengths of all edges may not decrease the area.
Moreover, the surface which minimize the area among all surfaces with fixed  triangulation might be not saddle;
try to construct such example.


%%%%%%%%%%%%%%%%%%%%%%%%%%%%%%%%%%%%%%%%%%%%%%%%%%
{
\begin{wrapfigure}{r}{22 mm}
\begin{lpic}[t(-7 mm),b(-2 mm),r(0 mm),l(0 mm)]{pics/Convex-triangulation(1)}
\end{lpic}
\end{wrapfigure}

\parbf{Coherent triangulation.} 
Look at the diagram and think.
\qeds

The problem was discussed by 
Israel Gelfand, 
Mikhail Kapranov 
and Andrei Zelevinsky in \cite[7C]{GKZ}.
The given example is closely related to so called \emph{Sch\"onhardt polyhedron}, an example of non-convex polyhedron which does not admit a triangulation \cite[see][]{schoenhardt}.

}

%%%%%%%%%%%%%%%%%%%%%%%%%%%%%%%%%%%%%%%%%%%%%%%%%%
\parbf{A sphere with one edge.}
An example, say $P$, can be found among the spherical polyhedral spaces which admit
an isometric $\mathbb{S}^1$-action with geodesic orbits.

Fix large relatively prime integers $p>q$. 
Consider the triangle $\Delta$ with angles $\tfrac\pi p$, $\tfrac\pi q$ and say $\pi\cdot(1-\tfrac1 p)$ in the sphere of radius $\tfrac12$.
Denote by $\hat \Delta$ the  doubling of $\Delta$ along  its boundary.
Note that $\hat \Delta$ is homeomorphic to $\mathbb S^2$,
it has 3 singular points with total angles $2\cdot\tfrac\pi p$,
$2\cdot\tfrac\pi q$ and $2\cdot\pi\cdot(1-\tfrac1 p)$.

Consider $\mathbb S^1$-action on $\mathbb S^3\subset\CC^2$ by the diagonal matrices $\left(\begin{smallmatrix}z^p&0\\0&z^q\end{smallmatrix}\right)$, $z\in\mathbb S^1\subset\CC$.
Construct a spherical polyhedral metric $\rho$ on  $\mathbb S^3$
such that the $\mathbb S^1$-orbits become geodesics 
and the quotient $(\mathbb S^3,\rho)/\mathbb S^1$
is isometric to $\hat \Delta$.

In the constructed example 
the singular points with total angles $2\cdot\tfrac\pi p$ and
$2\cdot\tfrac\pi q$
should correspond to the points with isotropy groups $\ZZ/p$ and $\ZZ/q$ of the action.
The points in $P=(\mathbb{S}^3,d)$ on the orbits over these points will be regular points of $P$.
The singular locus $P^\star$
of $P$ will be formed by the orbit corresponding to the remaining singular point of  $\hat \Delta$.
By construction,
\begin{itemize}
\item $P^\star$ is a closed geodesic with angle $2\cdot\pi\cdot(1-\tfrac1p)$ around it.
\item $P^\star$ forms a $(p,q)$-torus knot in the ambient $\mathbb{S}^3$.
\end{itemize}
\qedsf

The construction given by Dmitri Panov in \cite{panov-Kaeler}.
The cone $K$ over $P$ is a polyhedral space with natural complex structure;
that is, one can cut simplices from $\CC^2$ and the glue the cone from them in such a way that complex structures will agree along the gluings.
Moreover the cone $K$ can be holomorphically parametrized by $\CC^2$ in such a way that its singular set becomes an algebraic curve $z^p=w^q$ in some $(z,w)$-coordinates of $\CC^2$.

\begin{wrapfigure}[5]{r}{20 mm}
\begin{lpic}[t(-8 mm),b(-4 mm),r(0 mm),l(0 mm)]{pics/thurston(1)}
\end{lpic}
\end{wrapfigure}

It would be interesting to understand what types of knots can appear this way;
the given construction produces only torus knots.
We do not know if such knots exist for Euclidean polyhedral spaces, but there are links.
For example, the Borromean rings can appear as the singular set of a Euclidean polyhedral metrics on $\mathbb S^3$.
It can be obtained by gluing each face of cube to it self
along the reflections in the middle lines shown on the picture. 
This construction is due to William Thurston \cite[see][]{thurston}

%%%%%%%%%%%%%%%%%%%%%%%%%%%%%%%%%%%%%%%%%%%%%%%%%%
\parbf{Triangulation of a torus.}
Let us equip the torus with the flat metric such that each triangle is equilateral.
The metric will have two singular cone points,
the first corresponds to the vertex $v_5$ with 5 triangles,
the total angle around this point is $\tfrac53\cdot\pi$
and the second corresponds to the vertex $v_7$ with 7 triangles,
the total angle around this point is $\tfrac73\cdot\pi$.

Prove the following.

\parit{Observation.} The holonomy group of this metric is generated by rotation by $\tfrac\pi3$.

\medskip

Consider a closed geodesic $\gamma_1$ which minimize the length of all circles which are not null-homotopic.
Let $\gamma_2$ be an other closed geodesic which minimize the length and is not homotopic to any power of $\gamma_1$.

Show that $\gamma_1$ and $\gamma_2$ intersect at a single point.

Show that $\gamma_i$ cannot pass $v_5$.

Apply the observation above 
to show that 
if $\gamma_i$ pass through $v_7$,
then the measure  
of one of two angles which $\gamma_i$ cuts at $v_7$ equals to $\pi$.
Use the latter statement to show that  
one can push $\gamma_i$ aside so it does not longer pass through $v_7$, but remains a closed geodesic.

Cut $\TT^2$ along $\gamma_1$ and $\gamma_2$.
In the obtained quadrilateral, connect $v_5$ to $v_7$ by a minimizing geodesic and cut along it.
This way we obtain an annulus with flat metric.
Look at the neighborhood of the boundary components and show that the annulus can and cannot be isometrically immersed into the plane;
this is a contradiction.
\qeds

There are flat metrics on the torus with 
only two singular points 
which have the total angles $\tfrac53\cdot\pi$ and $\tfrac73\cdot\pi$.
Such example can be obtained by identifying the hexagon on the picture  according to the arrows.
But the holonomy group of the obtained torus is generated by the rotation by angle $\tfrac\pi6$. 
In particular, the observation is necessary in the proof.

\begin{wrapfigure}{r}{20 mm}
\begin{lpic}[t(-0 mm),b(-4 mm),r(0 mm),l(0 mm)]{pics/57-triangulation(1)}
\end{lpic}
\end{wrapfigure}

The same argument shows that 
holonomy group of flat torus with exactly two singular points with total angle $2\cdot(1\pm \tfrac1n)\cdot\pi$ has more than $n$ elements.
In the solution we did the case $n=6$.

The problem was originally discovered and solved by Stanislav Jendrol' %???\soft{l}
and Ernest Jucovi\v{c}, in \cite{jendrol-jucovich},
their proof is combinatorial.
The solution described above was given by Rostislav Matveyev
in his lectures \cite[see][]{matveyev}.
A complex-analytic proof was found by 
Ivan Izmestiev, 
Robert Kusner, 
G\"unter Rote, 
Boris Springborn 
and John Sullivan in \cite{izmestiev-rote-springborn-kusner}. 







%%%%%%%%%%%%%%%%%%%%%%%%%%%%%%%%%%%%%%%%%%%%%%%%%%
\parbf{No simple geodesics.}
The curvature of a vertex on the surface of a convex polyhedron
is defined as the $2\cdot\pi-\theta$, where $\theta$ is the total angle around the vertex.

Notice that a simple closed geodesic cuts the surface into two discs with total curvature $2\cdot\pi$ each.
Therefore it is sufficient to construct a convex polyhedron with curvatures of the vertices $\omega_1,\omega_2,\dots,\omega_n$ such that
$2\cdot\pi$ cannot be obtained as sum of some of $\omega_i$.
An example of that type can be found among 3-simplexes.
\qeds

The problem is due to Gregory Galperin \cite[see][]{galperin} 
and rediscovered by Dmitry Fuchs and Serge Tabachnikov \cite[see 20.8 in][]{fuchs-tabachnikov}.

