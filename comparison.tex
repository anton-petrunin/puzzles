\csname @openrightfalse\endcsname
\chapter{Comparison geometry}

In this chapter we consider Riemannian manifolds with curvature bounds.

This chapter is the very demanding;
we assume that the reader is familiar with the 
main definitions in the subject,
including Jacobi fields, 
Shape operator and second fundamental form, 
equations of Riccati and Jacobi,
comparison theorems and Morse theory.
The classical book \cite{cheeger-ebin} by Cheeger and Ebin covers all the  necessary  material.

%%%%%%%%%%%%%%%%%%%%%%%%%%%%%%%%%%%%%%%%%
\subsection*{Geodesic immersion\hard}

\begin{pr}{\hard}{Geodesic immersion}
\label{Geodesic immersion}
Let $M$ be a simply connected positively curved Riemannian manifold and $\iota\:N\looparrowright M$ be a totally geodesic immersion.
Assume that 
\[\dim N>\tfrac 12\cdot \dim M.\]
Prove that $\iota$ is an embedding.
\end{pr}

%%%%%%%%%%%%%%%%%%%%%%%%%%%%%%%%%%%%%%%%%%%%%%%%%%
\parit{Semisolution.}
Set $n=\dim N$, $m=\dim M$.

Fix a smooth increasing strictly concave function $\phi$.
Consider the function $f=\phi\circ\dist_N$.

Note that if $f$ is smooth at some point $x\in M$ 
then $\Hess_xf$, the Hessian of $f$ at $x$, 
has at least $n+1$ negative eigenvalues.

Moreover, at any point $x\notin \iota(N)$ the same holds in the barrier sense.
That is, there is a smooth function $h$ defined on $M$
such that $h(x)=f(x)$, $h(y)\ge f(y)$ for any $y$
and $\Hess_xh$ has at least $n+1$ negative eigenvalues.

Use that $m< 2\cdot n$ and the described property to prove the following
analog of Morse lemma for $f$.

\parbf{Claim.}
{\it Given $x\notin \iota(N)$ there is a neighborhood $U\ni x$ such that the set 
\[U_-=\set{z\in U}{f(z)<f(x)}\] is simply connected.}

\medskip

Since $M$ is simply connected,
any closed curve in $\iota(N)$
can be contracted by a disc, say $s_0\:\mathbb D\to M$.

Applying the claim, one can construct an $f$-decreasing homotopy starting at $s_0$.
That is,
there is an homotopy $s_t\:\mathbb D\to M$, $t\in [0,1]$ 
such that $s_t(\partial \mathbb D)\subset \iota(N)$ for any $t$ 
and $s_1(\mathbb D)\subset \iota(N)$.
It follows that $\iota(N)$ is simply connected.

Finally note that if $\iota\:N\to M$ has a self-intersection,
then the image
$\iota(N)$ is not simply connected.
Hence the result follows.\qeds


The statement was proved by 
Fuquan Fang, 
S\'ergio Mendon\c{c}a 
and Xiaochun Rong in \cite{FMR}.
The main idea was discovered by 
Burkhard Wilking \cite[see][]{wilking-2003}.

%%%%%%%%%%%%%%%%%%%%%%%%%%%%%%%%%%%%%%%%%
\subsection*{Geodesic hypersurface}
\begin{pr}{\easy}{Geodesic hypersurface}\label{Geodesic hypersurface}
Prove that 
if a compact connected positively curved manifold $M$ admits a totally geodesic embedded hypersurface,
then $M$ or its double cover is homeomorphic to the sphere.
\end{pr}

%%%%%%%%%%%%%%%%%%%%%%%%%%%%%%%%%%%%%%%%%
\subsection*{If convex, then embedded}

\begin{pr}{}{If convex, then embedded}\label{If convex then embedded} 
Let $M$ be a complete simply connected Riemannian manifold 
with non-positive curvature 
and dimension at least $3$.
Prove that any immersed locally convex
compact hypersurface in $M$ is embedded.
\end{pr}

%%%%%%%%%%%%%%%%%%%%%%%%%%%%%%%%%%%%%%%%%
\subsection*{Immersed ball\hard}

\begin{pr}{\hard}{Immersed ball}\label{Immersed ball}
Prove that any immersed locally convex
hypersurface $\iota\:\Sigma\looparrowright M$
in a compact positively curved manifold $M$ of dimension $m\ge 3$, is the boundary of an immersed ball. 
That is, there is an immersion of a closed ball $f\:\bar B^m\looparrowright M$ and a diffeomorphism $h\:\Sigma\to\partial \bar B^m$
such that $\iota=f\circ h$.
\end{pr}

%%%%%%%%%%%%%%%%%%%%%%%%%%%%%%%%%%%%%%%%%
\subsection*{Minimal surface in the sphere}

\label{minimal surface}
A  smooth $n$-dimensional surface $\Sigma$ in
an $m$-dimensional Riemannian manifold $M$ is called minimal
if it locally minimized the $n$-dimensional area;
that is, sufficiently small regions of $\Sigma$ do not admit area decreasing deformations with fixed boundary.

The minimal surfaces can be also defined via mean curvature vector as follows.
Let $\T=\T\,\Sigma$ and $\mathrm{N}=\mathrm{N}\,\Sigma$ correspondingly tangent and normal bundle.
Let $s$ denotes the second fundamental form of $\Sigma$;
it is a quadratic from on $\T$ with values in $\mathrm{N}$.
Let  $e_i$ is an orthonormal basis for $\T_x$, 
set 
$$H_x=\sum_i s(e_i,e_i)\in \mathrm{N}_x;$$ 
it is the mean curvature vector at $x\in \Sigma$. 
We say that $\Sigma$ is \index{minimal surface}\emph{minimal} if $H\equiv 0$.

\begin{pr}{}{Minimal surface in the sphere}\label{almgren} 
Let $\Sigma$ be a closed $n$-dimensional 
minimal surface
in $\mathbb{S}^m$.
Prove that
$\vol_n \Sigma\ge \vol_n \mathbb{S}^n$.
\end{pr}

%%%%%%%%%%%%%%%%%%%%%%%%%%%%%%%%%%%%%%%%%
\subsection*{Hypercurve}

The Riemannian curvature tensor $R$
can be viewed as an operator $\text{\bf R}$ on the space of tangent bi-vectors $\bigwedge^2 \T$;
it is uniquely defined by identity
$$\langle\mathbf{R}(X\wedge Y),V\wedge W\rangle
=
\langle R(X,Y)V,W\rangle,$$
The operator $\mathbf{R}\:\bigwedge^2 \T\to \bigwedge^2 \T$ is called \index{curvature operator}\emph{curvature operator} and it is said to be {}\emph{positive definite} if
$\langle\mathbf{R}(\phi),\phi\rangle>0$ for all non zero
bi-vector $\phi\in\bigwedge^2 \T$.


\begin{pr}{}{Hypercurve}\label{codim=2}
Let $M^m\hookrightarrow \RR^{m+2}$ be a closed smooth $m$-dimensional
submanifold and let  $g$ be the  induced Riemannian metric on $M^m$.
Assume that sectional curvature of $g$ is positive.
Prove that the curvature operator of $g$ is positive definite.
\end{pr}

The second fundamental form for manifolds of arbitrary codimension
which we are about to describe 
might help to solve this problem.

Assume $M$ is a smooth submanifold in $\RR^m$.
Given a point $p\in M$ denote by $\T_p$ and $\mathrm{N}_p=\T_p^\bot$
the tangent and normal spaces of $M$ at $p$.
The \index{second fundamental form}\emph{second fundamental form} of $M$ at $p$ is defined as $$s(X,Y)=(\nabla_X Y)^\bot,$$ 
where $(\nabla_X Y)^\bot$ denotes the orthogonal projection of covariant derivative $\nabla_X Y$ onto the normal bundle.

The curvature tensor of $M$ can be found from the second fundamental form using the following  formula
\[\langle R(X,Y)Y,X\rangle=\langle s(X,X),s(Y,Y)\rangle-\langle s(X,Y),s(X,Y)\rangle,\]
which is direct generalization of the formula for Gauss curvature of surface.


%%%%%%%%%%%%%%%%%%%%%%%%%%%%%%%%%%%%%%%%%
\subsection*{Horosphere}

We say that a Riemannian manifold has negatively pinched sectional curvature, if its sectional curvature at any point in any sectional direction lies in  $[-a^2, -b^2]$, for fixed constants $a>b>0$.

Let $M$ be a complete Riemannian manifold
and $\gamma$ is a ray in $M$; 
that is, $\gamma\:[0, \infty)\to M$ is a minimizing unit-speed geodesic.

The \index{Busemann function}\emph{Busemann function} $b_\gamma\:M\to\RR$ is defined by
$$b_\gamma(p)=\lim_{t\to\infty}\left(|p-\gamma(t)|_X-t\right).$$
From the triangle inequality, 
the expression under the limit is non-increasing in $t$; 
therefore  the limit above is defined for any $p$.

A \index{horosphere}\emph{horosphere} in $M$ is defined as a level set of a Busemann function
in $M$.

We say that a complete Riemannian manifold $M$ has \index{polynomial volume growth}\emph{polynomial volume growth} if for some (and therefore any) $p\in M$, 
we have 
$$\vol B(p,r)_M\z\le C\cdot (r^k+1),$$ 
where $B(p,r)_M$ is the ball in $M$ and  $C$, $k$ are real constants.

\begin{pr}{}{Horosphere}\label{Horosphere} Let $M$ be a complete simply connected manifold with negatively pin\-ched sectional curvature. 
And let $\Sigma\subset M$ be an horosphere in $M$.
 Prove that
$\Sigma$ with the induced intrinsic metric 
has polynomial volume growth.
\end{pr}

%%%%%%%%%%%%%%%%%%%%%%%%%%%%%%%%%%%%%%%%%
\subsection*{Minimal spheres}


Recall that two subsets $A$ and $B$ in a metric space $X$ are called \index{equidistant sets}\emph{equidistant} if the distance function $\dist_A\:X\to\RR$ is constant on $B$ and $\dist_B$ is constant on $A$.

The minimal surfaces are defined on page \pageref{minimal surface}.

\begin{pr}{}{Minimal spheres}\label{Minimal spheres}
Show that a 
$4$-dimensional
compact 
positively curved 
Riemannian manifold 
cannot contain infinite number of  mutually
 equidistant minimal 2-spheres.
\end{pr}


%%%%%%%%%%%%%%%%%%%%%%%%%%%%%%%%%%%%%%%%%
\subsection*{Positive curvature and symmetry\thm}

\begin{pr}{\thm}{Positive curvature and symmetry}\label{kleiner-hopf} 
Assume $\mathbb S^1$ acts isometrically on a $4$-dimensional positively curved closed Riemannian manifold.
Show that the action 
has at most $3$ isolated fixed points.
\end{pr}

Let us give some background material which can help to solve the problem. 

Let $X$ and $Y$ be metric spaces.
An onto map $f\:X\to Y$ is called \index{submetry}\emph{submetry} if it is short and 
for any $x\in X$ and $y\in Y$ there is $x'\in X$ such that 
\[f(x')=y\ \ \text{and}\ \  |x-x'|_X=|f(x)-y|_Y.\]
Here is a general statement required in the proof.

\begin{itemize}
\item If $f\:X\to Y$ is a  submetry
and $X$ is an Alexandrov space with curvature $\ge \kappa$, 
then $Y$ 
is also an Alexandrov space with curvature $\ge \kappa$.
\end{itemize}

Note that 
if  $G$ is a compact subgroup of isometries of a metric space $X$, 
then the quotient map $X\to X/G$ is a submetry,
assuming that the orbit space $X/G$ equipped with the Hausdorff metric on the orbits.

Complete Riemannian manifold with sectional curvature $\ge \kappa$ are Alexandrov spaces.

The following statement follows from the one above 
and it can be used directly in the solution.
\begin{itemize}
\item If $(M,g)$ is a Riemannian manifold with sectional curvature $\ge 1$ which admits a continuous isometric action of $\mathbb S^1$, 
then $(M,g)/\mathbb S^1$ is an Alexandrov space with curvature $\ge 1$.
\end{itemize}
For more on Alexandrov Geometry, see \cite{akp}.\qeds

%%%%%%%%%%%%%%%%%%%%%%%%%%%%%%%%%%%%%%%%%
\subsection*{Energy minimizer}

Let $F$ be a smooth map from a closed Riemannian manifold $M$ to a Riemannian manifold $N$.
Then energy functional of $F$ is defined as
\[E(F)=\int\limits_M |d_xF|^2\cdot d_x\vol_M.\]
If $(a_{i,j})$ denote the components 
of the differential $d_xF$ 
written in the orthonormal bases of the tangent spaces $\T_xM$ and $\T_{F(x)}N$,
then 
\[|d_xF|^2=\sum_{i,j}a_{i,j}^2.\]

\begin{pr}{}{Energy minimizer}\label{Energy minimizer}
Show that the identity map on $\RP^m$ is 
energy
minimizing in its homotopy class.
Here we assume that $\RP^m$ is equipped with canonical metric.
\end{pr}

%%%%%%%%%%%%%%%%%%%%%%%%%%%%%%%%%%%%%%%%%
\subsection*{Curvature vs. injectivity radius\thm}

\begin{pr}{\thm}{Curvature vs. injectivity radius}\label{scalar-curv} 
Let $(M,g)$ be a closed 
Riemannian $m$-dimensional manifold.
Assume average of sectional curvatures of $(M,g)$ is $1$. 
Show that the injectivity radius of $(M,g)$ is at most $\pi$.
\end{pr}

A solution use a corollary of Liouville's theorem 
which states that geodesic flow on the tangent bundle to a Riemannian manifold preserves the volume form.

%%%%%%%%%%%%%%%%%%%%%%%%%%%%%%%%%%%%%%%%%
\subsection*{Almost flat manifold}

\emph{Nil-manifolds} form the minimal class of manifolds which includes a point, and has the following property:  
the total space of any principle $\mathbb{S}^1$-bundle over a nil-manifold is a nil-manifold. 

Th nil-manifold can be also defined as the quotients of a connected nilpotent Lie group by a lattice.

A compact Riemannian manifold $M$ is called $\eps$-flat if its sectional curvature at all points in all directions lie in the interval $[-\eps,\eps]$. 

The main theorem of Gromov in \cite{gromov-almost-flat}, 
states that for any positive integer $n$ there is $\eps>0$ such that any $\eps$-flat compact $n$-dimensional manifold with diameter at most $1$ admits a finite cover by a nil-manifold.
A more detailed proof can be found in \cite{buser-karcher}
and a more precise statement can be found in \cite{ruh}.

\begin{pr}{}{Almost flat manifold}\label{almost-flat}
Given $\eps>0$ construct a compact Riemannian manifold $M$ of sufficiently large dimension which admits a Riemannian metric with diameter $\le 1$ and sectional
curvature $|K|<\eps$,
but does not admit a finite covering by a nil-manifold.
\end{pr}

%%%%%%%%%%%%%%%%%%%%%%%%%%%%%%%%%%%%%%%%%
\subsection*{Approximation of quotient}

\begin{pr}{}{}
Let $(M,g)$ be a compact Riemannian manifold 
and $G$ is a compact Lie group acting by isometries on $(M,g)$.
Construct a sequence of metrics $g_n$ on a fixed manifold $N$ such that $(N,g_n)$ converges to the quotient space $(M,g)/G$ in the sense of Gromov--Hausdorff.
\end{pr}


%%%%%%%%%%%%%%%%%%%%%%%%%%%%%%%%%%%%%%%%%
\subsection*{Polar points\many}

\begin{pr}{\many}{Polar points}\label{milka-polar} 
Let $M$ be a compact Riemannian manifold with sectional curvature $\ge 1$ and $\dim M\ge 2$. 
Prove that for any point $p\in M$ there is a point $p^*\in M$ such that 
\[|p-x|_M+|x-p^*|_M\le \pi\]
for any $x\in M$.
\end{pr}

%%%%%%%%%%%%%%%%%%%%%%%%%%%%%%%%%%%%%%%%%
\subsection*{Isometric section\hard}

\begin{pr}{\hard}{Isometric section}\label{Isometric section}
Let $M$ and $W$ be compact Riemannian manifolds,
$\dim W>\dim M$
and $s\:W\to M$ be a Riemannian submersion.
Assume that $W$ has positive sectional curvature.
Show that $s$ does not admit an isometric section;
that is, there is no isometric embedding $\iota\:M\hookrightarrow W$ such that $s\circ\iota(p)=p$ for any $p\in M$.
\end{pr}

%%%%%%%%%%%%%%%%%%%%%%%%%%%%%%%%%%%%%%%%%
\subsection*{Warped product}

Let $(M,g)$ and $(N,h)$ be Riemannian manifolds 
and $f$ be a smooth positive function defined on $M$.
Consider the product manifold $W\z=M\times N$.
Given a tangent vector 
$X\z\in \T_{(p,q)} W
\z=\T_p M\times \T_p N$ denote by 
$X_M\z\in \T M$ and $X_N\z\in \T N$ its projections.
Let us equip $W$ with the Riemannian metric defined as
\[s(X,Y)=g(X_M,Y_M)+f^2\cdot h(X_N,Y_N).\]
The obtained Riemannian manifold $(W,s)$ is called \index{warped product}\emph{warped product} of $M$ and $N$ with respect to $f\:M\to \RR$;
it can be written as  $(W,g)\z=(N,h)\times_f(M,g)$.

\begin{pr}{}{Warped product}\label{Warped product}
Assume $M$ is an oriented 3-dimensional Riemannian manifold with positive scalar curvature 
and $\Sigma\subset M$ is an oriented smooth hypersurface which is area minimizing in its homology class.

Show that there is a positive smooth function $f\:\Sigma\to \RR$
such that the warped product $\mathbb S^1\times_f \Sigma$
has positive scalar curvature;
here $\Sigma$ is equipped with the Riemannian metric
induced from $M$.
\end{pr}

%%%%%%%%%%%%%%%%%%%%%%%%%%%%%%%%%%%%%%%%%
\subsection*{No approximation\many}

\begin{pr}{\many}{No approximation}\label{No approximation}
Prove that 
if $p\not=2$,
then $\RR^m$ 
equipped with the metric induced by the $\ell^p$-norm 
cannot be a
Gromov--Hausdorff limit of
$m$-dimensional Riemannian manifolds $(M_n,g_n)$ with $\Ric_{g_n}\z\ge C$ for some fixed constant $C\in\RR$.
\end{pr}

%%%%%%%%%%%%%%%%%%%%%%%%%%%%%%%%%%%%%%%%%
\subsection*{Area of spheres}

\begin{pr}{}{Area of spheres}\label{Area of spheres}
Let $M$ be a complete non-compact Riemannian manifold 
with non-negative Ricci curvature and $p\in M$.
Then there is $\eps>0$ such that 
$$\area\left[\partial B(p,r)\right]>\eps$$
for all sufficiently large $r$.
\end{pr}

%%%%%%%%%%%%%%%%%%%%%%%%%%%%%%%%%%%%%%%%%
\subsection*{Curvature hollow}

\begin{pr}{}{Curvature hollow}\label{Curvature hollow}
Construct a Riemannian metric on $\RR^3$ 
which is Euclidean outside of an open bounded set $\Omega$ 
and with negative scalar curvature in $\Omega$.
\end{pr}

%%%%%%%%%%%%%%%%%%%%%%%%%%%%%%%%%%%%%%%%%
\subsection*{Flat coordinate planes}

\begin{pr}{}{Flat coordinate planes}\label{Flat coordinate planes}
Let $g$ be a Riemannian metric on $\RR^3$,
such that the coordinate planes $x=0$, $y=0$ and $z=0$ are flat and totally geodesic.
Assume the sectional curvature of $g$ is either non-negative or non-positive.
Show that in both cases $g$ is flat. 
\end{pr}

%%%%%%%%%%%%%%%%%%%%%%%%%%%%%%%%%%%%%%%%%
\subsection*{Two-convexity\many}

An open subset $V$ with smooth boundary in the Euclidean space  
is called \index{two-convex set}\emph{two-convex} if at most one principle curvatures in the outward direction to $V$ is negative.

Equivalently $V$ is two-convex if for any closed plane curve $\gamma$ in $V$ which is contactable in $V$ 
the plane region bounded by $\gamma$ belongs to $V$.

\begin{pr}{\many}{Two-convexity}\label{Two-convexity}
Let $K$ be a closed set bounded by a smooth surface
in $\RR^4$.
Assume $K$ contains two coordinate planes $$\{(x,y,0,0)\in\RR^4\}\ \ 
\text{and}
\ \ \{(0,0,z,t)\in\RR^4\}$$
in its interior 
and also belongs to the closed $1$-neighborhood of these two planes.

Show that the complement of $K$ cannot be two-convex.
\end{pr}

%%%%%%%%%%%%%%%%%%%%%%%%%%%%%%%%%%%%%%%%%%%%%%%%%%
\section*{Semisolutions}



%%%%%%%%%%%%%%%%%%%%%%%%%%%%%%%%%%%%%%%%%%%%%%%%%%
\parbf{Geodesic hypersurface.}
Assume $\Sigma$ is a totally geodesic embedded hypersurface in $M$.
Without loss of generality, we can assume that $\Sigma$ is connected.

The complement $M\backslash\Sigma$ has one or two connected components.
First let us show that if the number of connected components is two, 
then $M$ is homeomorphic to sphere.

By cutting $M$ along $\Sigma$ 
we get two manifolds, say $M_1$ and $M_2$,
with geodesic boundaries. 
Prove that the distance functions to the boundary 
$f_1\:M_1\to\mathbb{R}$ and $f_2\:M_2\to\mathbb{R}$ are strictly convex in the interiors of the manifolds.

Smooth the functions $f_i$ keeping them convex; 
this can be done by applying Greene--Wu Theorem \cite[see Thm. 2 in][]{greene-wu}.
In particular each $f_i$ has singe critical point which is its maximum.

Applying Morse lemma, we get that each manifold $M_i$ is homeomorphic to a ball; 
hence $M$ 
is homeomorphic to the sphere.

If $M\backslash\Sigma$ is connected,
passing to a double cover of $M$,
we reduce the problem to the case which already has been considered.\qeds

The problem was suggested by Peter Petersen.



%%%%%%%%%%%%%%%%%%%%%%%%%%%%%%%%%%%%%%%%%%%%%%%%%%
\parbf{If convex, then embedded.}
Observe first that any closed embedded locally convex hypersurface in a non-positively curved simply connected complete manifold bounds a convex region.


Let $\Sigma$ be an immersed locally convex hypersurface in $M$.
Set 
\[m=\dim \Sigma=\dim M-1\]

Given a point in $p$ on $\Sigma$ 
denote by $p_r$ the point on distance $r$ from $p$
which lies on the geodesic starting from $p$ in the outer normal direction to $\Sigma$.
For fixed $r\ge 0$,
the points $p_r$ sweep an immersed locally convex hypersurface which we denote by $\Sigma_r$.

Fix $z\in \Sigma$.
Denote by $S_r$ the sphere of radius $r$ centered at $z$.
Note that $S_r$ is diffeomorphic to $m$-dimensional sphere.

Denote by $d$ the diameter of $\Sigma$.
Note that for all $r>0$
any point on $\Sigma_r$
lies on a distance at most $d$ from $S_r$.
Conclude that for large $r$ the closest point projection $\phi_r\:\Sigma_r\to S_r$ is an immersion.


Since $\Sigma$ is connected
and $m\ge 2$, it follows that $\phi_r$ is a diffeomorphism for all large $r$.

By the observation above, $\Sigma_r$ bounds a convex region for all large $r$.
By an open-closed argument, the same holds for all $r\ge 0$.
Hence the result follows.\qeds

The problem is due to Stephanie Alexander \cite[see][]{alexander}.



%%%%%%%%%%%%%%%%%%%%%%%%%%%%%%%%%%%%%%%%%%%%%%%%%%
\parbf{Immersed ball.}
Equip $\Sigma$ with the induced intrinsic metric.
Denote by $\kappa$ the lower bound for principle curvatures of $\Sigma$.
Note that we can assume that $\kappa>0$.

Fix sufficiently small $\eps=\eps(M,\kappa)>0$.
Given $p\in \Sigma$ consider the lift $\tilde h_p\:B(p,\eps)\to \T_{h(p)}$ along the exponential map $\exp_{h(p)}\:\T_{h(p)}\to M$.
More precisely:
\begin{enumerate}
\item Connect each point $q\in B(p,\eps)\subset \Sigma$ to $p$
by a minimizing geodesic  path $\gamma_q\:[0,1]\to \Sigma$
\item Consider the lifting $\tilde\gamma_q$ in $\T_{h(p)}$; 
that is the curve such that $\tilde\gamma_q(0)=0$ and $\exp_{h(p)}\circ\tilde\gamma_q(t)=\gamma_q(t)$ for any $t\in[0,1]$.
 \item Set $\tilde h(q)=\tilde\gamma_q(1)$.
\end{enumerate}

Show that all the hypersurfaces $\tilde h_p(B(p,\eps))\subset \T_{h(p)}$ has principle curvatures at least $\tfrac\kappa2$.

Use the same idea as in the solution of ``Immersed surface'' [page ~\pageref{Immersed surface}] to show that 
one can fix $\delta\z=\delta(M,\kappa)>0$ such that the restriction of $\tilde h_p|_{B(p,\delta)}$ is injective.
Conclude that the restriction $h|_{B(p,\delta)}$ is injective for any $p\in\Sigma$.

Now consider locally equidistant surfaces $\Sigma_t$ in the inward direction for small $t$. 
The principle curvatures of $\Sigma_t$ remain at least $\kappa$ in the barrier sense.
By the same argument as above, any $\delta$-ball in $\Sigma_t$
is embedded.

Applying open-closed argument we get a one parameter family of locally convex locally equidistant surfaces $\Sigma_t$
for defined in a maximal interval $[0,a)$
and 
the surface $\Sigma_a$ degenerates to a point, say $p$. 

To construct the immersion $\partial \bar B^m\looparrowright M$,
take the point $p$ as the image of the center $\bar B^m$ 
and take the surfaces $\Sigma_t$ as the restrictions of the  embedding to the spheres;
the existence of the immersion follows from the Morse lemma.\qeds

\begin{wrapfigure}[5]{r}{20 mm}
\begin{lpic}[t(-5 mm),b(0 mm),r(0 mm),l(0 mm)]{pics/ass(1)}
\end{lpic}
\end{wrapfigure}

As you see from the picture, 
the analogous statement does not hold in the two-dimensional case.

The proof presented above was indicated in the lectures of Mikhael Gromov \cite[see][]{gromov-SGMC};
the proof was written rigorously by Jost Eschenburg in \cite{eschenburg}.

A variation of Gromov's proof 
was obtained independently by Ben Andrews in \cite{andrews}.
Instead of equidistant deformation, 
he uses a so called \index{inverse mean curvature flow}\emph{inverse mean curvature flow};
this way he has to perform some calculations to show that convexity survives in the flow, 
but he does not have to worry about non-smoothness of the hypersurfaces. 




%%%%%%%%%%%%%%%%%%%%%%%%%%%%%%%%%%%%%%%%%%%%%%%%%%
\parbf{Minimal surface in the sphere.}
Fix a  geodesic $n$-dimensional sphere $\mathbb{S}^n$ in $\mathbb{S}^m$.

Given $r\in (0,\tfrac\pi2]$,
denote by $U_r$ and $\tilde U_r$ the tubular $r$-neighborhood 
of $\Sigma$ and $\mathbb{S}^n$ in $\mathbb{S}^m$ correspondingly.

Prove that $U_{\frac\pi2}\supset\mathbb{S}^m$.
Then it follows that
\[U_{\frac\pi2}=\tilde U_{\frac\pi2}=\mathbb{S}^m.
\leqno({*})\]

Prove that for any $x\in \partial U_r$ we have
\[H_r(x)\ge \tilde H_r,\] 
where $H_r(x)$ denotes the mean curvature of $\partial U_r$  at a point $x$
and $\tilde H_r$ is the mean curvature of $\partial\tilde U_r$.

Set 
\begin{align*}
a(r)&=\vol_{m-1} \partial U_r,
&
\tilde a(r)&=\vol_{m-1} \partial\tilde U_r,
\\
v(r)&=\vol_m U_r,
&
\tilde v(r)&=\vol_m \tilde U_r.
\intertext{by the coarea formula,}
\tfrac d{dr} v(r)&\aall a(r),
&
\tfrac d{dr}\tilde v(r)&=\tilde a(r).
\end{align*}
Note that
\begin{align*}\tfrac d{dr}a(r)&\le \int\limits_{\partial U_r} H_r(x)\cdot d_x\vol_{m-1}\le
\\
&\le a(r)\cdot \tilde H_r
\end{align*}
and
\begin{align*}
\tfrac d{dr}\tilde a(r)
&= \tilde a(r)\cdot \tilde H_r.
\intertext{It follows that}
\frac {v''(r)}{v(r)}&\le \frac {\tilde v''(r)}{\tilde v(r)}
\end{align*}
for almost all $r$. 
Therefore
\[v(r)\le\frac{\area\Sigma}{\area \mathbb{S}^n}\cdot \tilde v(r)\]

for any $r>0$.

According to $({*})$,
\[v(\tfrac\pi2)=\tilde v(\tfrac\pi2)=\vol\mathbb{S}^m.\]
Whence the result follows.\qeds

This problem is the most geometric part of the isoperimetric inequality proved by Frederick Almgren in \cite{almgren}.
The argument is similar to 
the proof of isometric inequality for manifolds with positive Ricci curvature
given by Mikhael Gromov in \cite{gromov-apendix}.

%%%%%%%%%%%%%%%%%%%%%%%%%%%%%%%%%%%%%%%%%%%%%%%%%%
\parbf{Hypercurve.}
Fix $p\in M$.
Denote by $s$ 
the second fundamental form of $M$ at $p$;
it is a symmetric bi-linear form on the tangent space $\T_pM$ of $M$ with values in the normal space $\mathrm{N}_pM$ to $M$.
Note that the normal space $\mathrm{N}_pM$ is two-dimensional.

Prove that if the sectional curvature of $M$ is positive, 
then
\[\<s(X,X),s(Y,Y)\> > 0\leqno({*})\]
for any pair of nonzero vectors $X,Y\in\T_pM$.

Show that $({*})$ implies that there is an orthonormal basis $e_1,e_2$ in $\mathrm{N}_pM$ 
such that the real-valued quadratic forms 
\begin{align*}
s_1(X,X)&=\<s(X,X),e_1\>,
&
s_2(X,X)&=\<s(X,X),e_2\>
\end{align*}
are positive definite.

Note that the curvature operators $R_1$ and $R_2$, 
defined by the following identity
\[R_{i}(X\wedge Y), V\wedge W\rangle 
=s_i(X,W)\cdot s_i(Y,V)-s_i(X,V)\cdot s_i(Y,W),\]
 are positive.
Finally, note that $R_{1}+R_{2}$ is the curvature operator of $M$ at $p$.\qeds

The problem is due to Alan Weinstein \cite[see][]{weinstein}.
Note that from \cite{micallef-moore}/\cite{boehm-wilking} it follows that
that the universal cover of $M$ is homeomorphic/diffeomorphic to a standard sphere.



%%%%%%%%%%%%%%%%%%%%%%%%%%%%%%%%%%%%%%%%%%%%%%%%%%
\parbf{Horosphere.}
Set 
$m=\dim \Sigma=\dim M-1$.

Let $b\:M\to\RR$ be the Busemann function such that $\Sigma=b^{-1}(\{0\})$.
Set  $\Sigma_r=b^{-1}(\{r\})$, so $\Sigma_0=\Sigma$.

Let us equip each $\Sigma_r$ with induced Riemannian metric.
Note that all $\Sigma_r$ have bounded curvature.
In particular, the unit ball in $\Sigma_r$ has volume bounded above by a universal constant, say $v_0$.
 

Given $x\in \Sigma$ denote by $\gamma_x$ 
the unit-speed geodesic
such that $\gamma_x(0)=x$ and $b(\gamma_x(t))=t$ for any $t$.
Consider the map $\phi_{r}\:\Sigma\to\Sigma_r$ defined as
$\phi_r\:x\mapsto \gamma_x(r)$.

Notice that $\phi_r$ is a bi-Lipschitz map with the Lipschitz constants $e^{a\cdot r}$ and $e^{b\cdot r}$.
In particular, the ball of radius $R$ in $\Sigma$ is mapped by $\phi_r$
to a ball of radius $e^{a\cdot r}\cdot R$ in $\Sigma_r$.
Therefore
\[\vol_m B(x,R)_\Sigma\le e^{m\cdot b\cdot r}\cdot \vol_m B(x,e^{a\cdot r}\cdot R)_{\Sigma_r}\]
for any $R,r>0$.
Taking $e^{a\cdot r}\cdot R=1$, we get
\[\vol_m B(x,R)_\Sigma\le v_0\cdot R^{m\cdot \frac ba}.\]
\qedsf

The problem was suggested by Vitali Kapovitch.

There are examples of horospheres as above with degree of polynomial growth higher than $m$.
For example, consider the horosphere $\Sigma$ in the
the complex hyperbolic space 
of real dimension $4$.
Clearly $m=\dim \Sigma=3$, but the degree of its volume growth is $4$.

In this case $\Sigma$ is isometric to the Heisenberg group defined below Twith a left-invariant metric.
It is a good exercise to show that any such metric has volume  growth of degree $4$.

Heisenberg group
is the group of $3\times3$ upper triangular matrices of the form
\[\begin{pmatrix}
 1 & a & c\\
 0 & 1 & b\\
 0 & 0 & 1\\
\end{pmatrix}\]
under the operation of matrix multiplication. 
                                                      



%%%%%%%%%%%%%%%%%%%%%%%%%%%%%%%%%%%%%%%%%%%%%%%%%%
\parbf{Minimal spheres.}
Choose a pair of sufficiently close minimal spheres $\Sigma$ and $\Sigma'$,
say assume that the distance $a$ between $\Sigma$ and $\Sigma'$ is strictly smaller than the injectivity radius of the manifold.
Note that in this case there is a bijection $\Sigma\to \Sigma'$, which will be denoted by $p\mapsto p'$ such that the distance $|p-p'|=a$ for any $p\in\Sigma$.

Let $\iota_p\:\T_p\to\T_{p'}$ be the parallel translation along the (necessary unique) minimizing geodesic from $p$ to $p'$.
Use the hairy ball theorem 
to show that there is a pair $(p,p')$ such that $\iota_p(\T_p\Sigma)=\T_{p'}\Sigma'$.

Consider pairs of unit-speed geodesics $\alpha$ and $\alpha'$ 
in $\Sigma$ and $\Sigma'$  
which start at $p$ and $p'$ correspondingly
and go in the parallel directions, say $\nu$ and $\nu'$. 
Set $\ell_\nu(t)=|\alpha(t)-\alpha'(t)|$.

Use the second variation formula to show that $\ell_\nu''(0)$ has negative average for all tangent directions $\nu$ to $\Sigma$ at $p$. 
In particular $\ell_\nu''(0)<0$ for a pair $\alpha$ and $\alpha'$ as above.
It follows that there are points $v\in\Sigma$ near $p$ 
and $v'\in\Sigma'$ near $p'$
such that 
\[|v-v'|<|p-p'|;\]
the latter leads to a contradiction.\qeds

It seems pleasurable that a 
compact 
positively curved 
4-dimensional manifold
cannot contain a pair of equidistant spheres.
The argument above implies that the distance between such a pair has to exceed the injectivity radius of the manifold.

{\sloppy 
The problem was suggested by Dmitri Burago.
Here is a short list of classical problems with similar solutions:
\begin{itemize}
\item Synge's problem \cite[see][]{synge}.
\begin{itemize}
 \item {\it Any compact even-dimensional orientable manifold with positive sectional curvature is
simply connected.}
\end{itemize}
\item Frankel's problems \cite[see][]{frankel}.
\begin{itemize}
\item {\it Any two compact minimal hypersurfaces in a Riemannian manifold with positive Ricci curvature must intersect.}
\item {\it Assume $\Sigma_1$ and $\Sigma_2$ be two compact geodesic submanifolds in a manifold with positive sectional curvature $M$ and 
\[\dim \Sigma_1+\dim \Sigma_2\ge \dim M.\] 
Show that $\Sigma_1\cap\Sigma_2\ne\emptyset$.}
\end{itemize}
\item Bochner's problem \cite[see][]{bochner}.
\begin{itemize}
\item{\it  Let $(M,g)$ be a closed Riemannian manifold with negative Ricci curvature.
Prove that $(M,g)$ does not admit an isometric $\mathbb{S}^1$-action.}
\end{itemize}
\end{itemize}
The problem ``Geodesic immersion'' on page \pageref{Geodesic immersion} can be considered as further development of the same idea.

}




%%%%%%%%%%%%%%%%%%%%%%%%%%%%%%%%%%%%%%%%%%%%%%%%%%
\parbf{Positive curvature and symmetry.}
Let $M$ be a 4-dimensional Riemannian manifold with isometric $\mathbb{S}^1$-action.
Consider the quotient space $X=M/\mathbb{S}^1$.
Note that $X$ is a positively curved 3-dimensional Alexandrov space.
In particular the angle $\measuredangle\hinge xyz$ between any two geodesics $[xy]$ and $[xz]$ is defined
and for any non-degenerate triangle $[xyz]$ 
formed by the minimizing geodesics $[xy]$, $[yz]$ and $[zx]$  in $X$ we have
\[\measuredangle\hinge xyz+\measuredangle\hinge yzx+\measuredangle\hinge zxy> \pi.
\leqno({*})\]

Assume $p\in X$ corresponds to a fixed point of $\mathbb{S}^1$-action.
Show that 
for any three geodesics $[px]$, $[py]$ and $[pz]$ in $X$ we have
\[\measuredangle\hinge pxy+\measuredangle\hinge pyz+\measuredangle\hinge pzx\le \pi.\leqno({*}{*})\]
and
\[\measuredangle\hinge pxy, \measuredangle\hinge pyz, \measuredangle\hinge pzx\le \tfrac\pi2.\leqno({*}{*}{*})\]

Arguing by contradiction,
assume that there are 4 fixed points $q_1$, $q_2$, $q_3$ and $q_4$.
Connect each pair $q_i\ne q_j$ by a minimizing geodesic $[q_iq_j]$.

Denote by $\omega$ the sum of all 12 angles of the type  $\measuredangle\hinge{q_i}{q_j}{q_k}$.
By $({*}{*}{*})$, each triangle $\triangle q_iq_jq_k$ is non-degenerate.
Therefore by $({*})$, we have
\[\omega>4\cdot\pi.\]
Applying $({*}{*})$ at each vertex $q_i$, we have 
\[\omega\le 4\cdot\pi,\]
a contradiction.\qeds


The problem is due to 
Wu-Yi Hsiang 
and Bruce Kleiner 
\cite[see][]{hsiang-kleiner}.
The connection of this proof to Alexandrov geometry was noticed by Karsten Grove in \cite{grove}.
An interesting new twist of this idea 
is given by 
Karsten Grove 
and Burkhard Wilking 
in  \cite{grove-wilking}.

%%%%%%%%%%%%%%%%%%%%%%%%%%%%%%%%%%%%%%%%%%%%%%%%%%
\parbf{Energy minimizer.}
Denote by $\mathcal{U}$ the unit tangent bundle over $\RP^m$
and $\mathcal{L}$ the space of projective lines in $\ell\:\RP^1\to \RP^m$.
The spaces $\mathcal{U}$ and $\mathcal{L}$ 
have dimensions $2\cdot m-1$ 
and $2\cdot(m-1)$
correspondingly.


According to Liouville's theorem, the identity
\[\int\limits_{\mathcal{U}}f(v)\cdot d_v\vol_{2\cdot m-1}
=
\int\limits_{\mathcal{L}}d_\ell\vol_{2\cdot(m-1)}\cdot\int\limits_{\RP^1} f(\ell'(t))\cdot dt\]
holds for any integrable function $f\:\mathcal{U}\to\RR$.

Let $F\:\RP^m\to\RP^m$ be a smooth map.
Note that up to a multiplicative constant,
the energy of $F$ can be expressed the following way
\[\int\limits_{\mathcal{U}} |dF(v)|^2\cdot d_v\vol_{2m-1}
=
\int\limits_{\mathcal{L}}d_\ell\vol_{2\cdot(m-1)}\cdot\int\limits_{\RP^1} |[d(F\circ \ell)](t)|^2\cdot dt.\]

The result follows since
\[\int\limits_{\RP^1} |[d(F\circ \ell)](t)|^2\cdot dt\ge \pi\]
for any line $\ell\:\RP^1\to \RP^m$.\qeds


The problem is due to Christopher Croke \cite[see][]{croke-energy}. 
He uses the same idea to show that the identity map on $\CP^m$ is energy minimizing in its homotopy class.
For $\mathbb S^m$, an analogous statement does not hold if $m\ge 3$.
In fact, 
if a closed Riemannian manifold $M$ 
has dimension at least 3 
and $\pi_1M=\pi_2M=0$,
then the identity map on $M$ is homotopic 
to a map with arbitrary small energy;
the latter was shown by Brian White in \cite{white}.

The same idea is used to prove Loewner's inequality on the volume in of $\RP^m$ with metric conformally equivalent to the canonical one \cite[see][]{gromov-filling}.



%%%%%%%%%%%%%%%%%%%%%%%%%%%%%%%%%%%%%%%%%%%%%%%%%%
\parbf{Curvature vs. injectivity radius.}
We will show that 
if the injectivity radius of the manifold $(M,g)$ is at least $\pi$,
then the average of sectional curvatures on $(M,g)$ is at most $1$.
This is equivalent to the problem.

Fix a point $p\in M$ and two orthonormal vectors $U,V\in\T_p M$.
Consider the geodesic $\gamma$ in $M$ such that $\dot\gamma(0)=U$.

Set $U_t=\dot\gamma(t)\in \T_{\gamma(t)}$ 
and let $V_t\in \T_{\gamma(t)}$ be the parallel translation of $V=V_0$ along $\gamma$.


Consider the field $W_t=\sin t\cdot V_t$ on $\gamma$.
Set 
\begin{align*}
\gamma_\tau(t)&=\exp_{\gamma(t)} (\tau\cdot W_t),
\\
\ell(\tau)&=\length(\gamma_\tau|_{[0,\pi]}),
\\
q(U,V)&=\ell''(0).
\end{align*}
Note that
$$q(U,V)
=
\int\limits_{0}^\pi [(\cos t)^2-K(U_t,V_t)\cdot (\sin t)^2]\cdot dt,
\leqno({*})$$
where $K(U,V)$ is the sectional curvature 
in the direction spanned by $U$ and $V$. 

Since any geodesics of length $\pi$ is minimizing,
we get $q(U,V)\ge0$ for any pair of orthonormal vectors $U$ and $V$.
It follows that average value of the right hand side in $({*})$ is non-negative.

By Liouville's theorem, while taking the average of $({*})$, we can switch the order of integrals;
therefore  
\[0\le \tfrac\pi2\cdot(1-\bar{K}),\]
where $\bar{K}$ denotes the average of sectional curvatures on $(M,g)$.
Hence the result follows.\qeds

Liouville's theorem has a number of similar applications,
one of the most beautiful is the sharp isoperimetric inequality for 
4-dimensional Hadamard manifolds;
it was proved by Christopher Croke in \cite{croke-4d},
see also \cite{croke-eigenvalue}.






%%%%%%%%%%%%%%%%%%%%%%%%%%%%%%%%%%%%%%%%%%%%%%%%%%
\parbf{Almost flat manifold.}
First prove that for given $\eps>0$, 
there is big enough $m$ and $m\times m$ integer matrix 
$A$ such that all its eigenvalues are $\eps$-close to $1$. 

Consider $(m+1)$-dimensional manifold $S$ obtained from $\TT^m\times [0,1]$ by gluing $\TT^m\times 0$ to $\TT^m\times 1$ along the map given by $A$.

Show that $S$ does not admit a finite cover by a nill-manifold.

Assuming that $\eps$ is small,
show that $S$ admits a metric with curvature and diameter sufficiently small.\qeds

\label{page-sol:almost-flat}
This example was constructed 
by Galina Guzhvina in \cite{guzhvina}.

It is expected that for small enough $\eps>0$,
a Riemannian manifold $(M,g)$ of any dimension 
with  $\diam(M,g)\le 1$ and $|K_g|\le \eps$ cannot be simply connected,
here $K_g$ denotes the sectional curvature of $g$.
It is not true if instead one asks only have $K_g\le \eps$;
in fact, 
for any $\eps>0$,
there are metrics $g$ on $\mathbb{S}^3$ 
with $K_g\le \eps$ and $\diam(\mathbb{S}^3,g)\le 1$; 
this example was originally constructed by Mikhael Gromov in \cite{gromov-almost-flat}; 
a simplified proof was given by 
Peter Buser
and Detlef Gromoll in \cite{buser-gromoll}.


%%%%%%%%%%%%%%%%%%%%%%%%%%%%%%%%%%%%%%%%%%%%%%%%%%
\parbf{Approximation of quotient.}
Note that $G$ admits an embedding into a compact connected Lie group $H$, say we can assume $H=\SO(n)$, for large enough $n$.

Fix a $\kappa\le 0$ such that the curvature bound of $(M,g)$ is bounded below by $\kappa$.

The bi-invariant metric $h$ on $H$ is nonnegatively curved.
Therefore for any positive integer $n$ the product $(H,\tfrac1n\cdot h)\times (M,g)$ is a Riemannina manifold with  curvature bounded below by $\kappa$.

The diagonal action of $G$ on $(H,\tfrac1n\cdot h)\times (M,g)$ is        isometric and free. 
Therefore 
the quotient $(H,\tfrac1n\cdot h)\times (M,g)/G$
is a Riemannian manifold, say $(N,g_n)$.
By O'Nail's formula, $(N,g_n)$ has curvature bounded below by $\kappa$.

Clearly, $(N,g_n)$ converge to $(M,g)/G$ as $n\to \infty$.\qeds

Some times this construction is called \index{Cheeger's trick}\emph{Cheeger's trick},
although it was used before Cheeger.
The earliest use of this construction  
I found \cite{GKM}, where it was used to show that Berger's spheres have positive curvature.
This trick is used to construct most of the known examples of positively and non-negatively curved manifolds
 \cite[see][]{cheeger,aloff-wallach,gromoll-meyer,eschenburg-spaces,bazajkin}.
 
The quotient space  $(M,g)/G$ has dinite dimension and curvature bounded below in the sense of Alexandrov. 
It is expected that not all spaces with this property admit approximation by Riemannina manifolds with curvature bounded below,
some partial results are discussed in \cite{pwz,kapovitch}.








%%%%%%%%%%%%%%%%%%%%%%%%%%%%%%%%%%%%%%%%%%%%%%%%%%
\parbf{Polar points.}
Fix a unit-speed geodesic $\gamma$ such that $\gamma(0)=p$.
Set $p^*=\gamma(\pi)$.

Prove that $p^*$ is a solution.\qeds

\parit{Alternative proof.} 
Assume the contrary;
that is, for any $x\in M$ there is a point $x'$ such that 
\[|x-x'|_g+|p-x'|_g>\pi.\]

Show that there is a continuous map $x\mapsto x'$
such that the above inequality holds for any $x$.

Fix sufficiently small $\eps>0$.
Prove that the set $W_\eps=M\backslash B(p,\eps)$ 
is homeomorphic to a ball 
and the map $x\mapsto x'$ sends $W_\eps$ into itself.

By Brouwer's fixed-point theorem, $x=x'$ for some $x$.
In this case 
\[|x-x'|_g+|p-x'|_g\le \pi,\]
a contradiction.\qeds
 
The problem is due to Anatoliy Milka \cite[see][]{milka-poly}.





%%%%%%%%%%%%%%%%%%%%%%%%%%%%%%%%%%%%%%%%%%%%%%%%%%
\parbf{Isometric section.}
Arguing by contradiction, 
assume $\iota\: M\z\to W$ is an isometric section.
It makes possible to treat $M$ as a submanifold in $W$.

Given $p\in M$,
denote by $\mathrm{N}^1_pM$ the sphere of unit normal vectors to $M$ at $p$.
Given $v\in \mathrm{N}^1_p$ and real value $k$,
set 
\[p^{k\cdot v}=s\circ\exp_{p} (k\cdot v).\]
Note that 
\[p^{0\cdot v}=p\ \ \text{for any}\ \  p\in M\ \ \text{and}\ \ v\in \mathrm{N}^1_p.\leqno({*})\]

Fix sufficiently small $\delta>0$.
By Rauch comparison, 
if $w\in \mathrm{N}^1_q$ 
is the parallel translation of $v\in \mathrm{N}^1_q$ 
along a minimizing geodesic from $p$ to $q$ in $M$,
then 
\[|p^{k\cdot v}-q^{k\cdot w}|_M<|p-q|_M
\leqno({*}{*})\]
assuming $|k|\le \delta$.
The same comparison implies that 
\[|p^{k\cdot v}-q^{k'\cdot w}|_M^2<|p-q|_M^2+ (k-k')^2
\leqno({*}{*}{*})\]
assuming $|k|,|k'|\le \delta$.

Choose $p$ and $v \in \mathrm{N}^1_p$ so that $r=|p-p^{\delta\cdot v}|$ 
takes the maximal possible value.
From $({*}{*})$ it follows that $r>0$.

Let $\gamma$ be the extension of the unit-speed minimizing geodesic from $p_v$ to $p$;
denote by $v_t$ the parallel translation of $v$ to $\gamma(t)$ along $\gamma$. 

We can choose the parameter of $\gamma$ so that $p=\gamma(0)$, $p^v=\gamma(-r)$.
Set $p_n=\gamma(n\cdot r)$, so $p=p_0$ and $p^v=p_{-1}$. 
Fix large integer $N$ and set $w_n=(1-\tfrac nN)\cdot v_{n\cdot r}$
and $q_n=p_n^{w_n}$.


\begin{center}
\begin{lpic}[t(-0 mm),b(0 mm),r(0 mm),l(0 mm)]{pics/perelman(1)}
\lbl[trw]{8,5;$p_{-1}{=}q_0$}
\lbl[tr]{14,5;$p_0$}
\lbl[tl]{15,5;$q_1$}
\lbl[tr]{24,5;$p_1$}
\lbl[tl]{26,5;$q_2$}
\lbl[tr]{34.5,5;$p_2$}
\lbl[tl]{36.5,5;$q_3$}
\lbl[tr]{45,5;$p_3$}
\lbl[tl]{47.2,5;$q_4$}
\lbl[tr]{55.5,5;$p_4$}
\lbl[b]{64,8;$M$}
\lbl[t]{64,5;$\dots$}
\lbl[b]{35.2,26,90;{\small $\exp_{p_3} (w_3)$}}
\end{lpic}
\end{center}


From $({*}{*}{*})$, there is a constant $C$ independent of $N$ such that
\[|q_k-q_{k+1}|<r+\tfrac C{N^2}\cdot\delta^2.\]
Therefore 
\[|q_{k+1}-p_{k+1}|>|q_k-p_k|-\tfrac C{N^2}\cdot\delta^2.\]
By induction, we get 
\[|q_N-p_N|>r-\tfrac C{N}\cdot\delta^2.\]
Since $N$ is large we get
\[|q_N-p_N|>0.\]
By $({*})$ we get $q_N=p_N^0=p_N$, a contradiction.\qeds


This proof is the core of the solution of Soul conjecture
by Grigori Perelman \cite[see][]{perelman}.

%%%%%%%%%%%%%%%%%%%%%%%%%%%%%%%%%%%%%%%%%%%%%%%%%%
\parbf{Warped product.}
Given $x\in \Sigma$, denote by $\nu_x$ the normal vector to $\Sigma$ at $x$ which agrees with the orientations of $\Sigma$ and $M$.
Denote by $\kappa_x$ the non-negative principle curvature of $\Sigma$ at $x$;
since $\Sigma$ is minimal the other principle curvature has to be $-\kappa_x$.

Consider the warped product $W=\mathbb S^1\times_f\Sigma$ for some positive smooth function $f\:\Sigma\to \RR$.
Assume that a point $y\in W$ projects to a point $x\in\Sigma$.
Straightforward computations show that
\begin{align*}
\Sc_W(y)
&=\Sc_\Sigma(x)-2\cdot\frac{\Delta f(x)}{f(x)}=
\\
&=\Sc_M(x)-2\cdot\Ric(\nu_x)-2\cdot\kappa_x^2-2\cdot\frac{\Delta f(x)}{f(x)}.
\end{align*}
Consider linear operator $L$ on the space of smooth functions on $\Sigma$ defined as 
\[(Lf)(x)= -[\Ric(\nu_x)+\kappa_x^2]\cdot f(x)-(\Delta f)(x)\]
It is sufficient to find a smooth function $f$ on $\Sigma$ such that
\[f(x)>0 \ \ \text{and}\ \ (Lf)(x)\ge 0\leqno({*})\]
for any $x\in \Sigma$.


Fix a smooth function $f\:\Sigma\to \RR$.
Extend the field $f(x)\cdot\nu_x$
on $\Sigma$ to a smooth field, say $v$, on whole $M$.
Denote by $\iota_t$ the flow along $v$ for time $t$ and set $\Sigma_t=\iota_t(\Sigma)$.

\parit{Informal end of proof.}
Denote by $H_t(x)$ the mean curvature of $\Sigma_t$ at $\iota_t(x)$.
Note that the value $(Lf)(x)$ is the derivative of
the function $t\mapsto H_t(x)$  at $t=0$.

Therefore the condition $({*})$
means that we can push $\Sigma$ into one of its sides 
so that its mean curvature does not increase in the first order.
Since $\Sigma$ is area minimizing,
the existence of such push follows;
read further if you are not convinced.\qeds

\parit{Formal end of proof.}
Denote by $\delta(f)$ the second variation of area of $\Sigma_t$;
that is consider the area function $a(t)=\area\Sigma_t$ 
and set $\delta(f)=a''(0)$.
Direct calculations show that
\begin{align*}
\delta(f)
&=
\int\limits_{\Sigma} 
\left(-[\Ric(\nu_x)+\kappa_x^2]\cdot f^2(x)+|\nabla f(x)|^2\right)\cdot d_x\area=
\\
&=\int\limits_{\Sigma} 
(Lf)(x)\cdot f(x)\cdot d_x\area.\end{align*}
Since $\Sigma$ is area minimizing we get 
\[\delta(f)\ge 0\leqno({*}{*})\] for any $f$.

Choose a function $f$ which minimize $\delta(f)$ among all the functions such that $\int_\Sigma f^2(x)\cdot d_x\area=1$.
Note that $f$ an eigenfunction 
for the linear operator $L$;
in particular $f$ is smooth.
Denote by $\lambda$ the eigenvalue of $f$;
by $({*}{*})$,
$\lambda\ge 0$.

Show that $f(x)>0$ at any $x$.
Since $Lf=\lambda\cdot f$, the inequalities $({*})$ follow.\qeds


The problem is due to Mikhael Gromov and Blaine Lawson \cite[see][]{gromov-lawson}.
Earlier, Shing-Tung  Yau and Richard Schoen showed that the same assumptions 
imply existence of conformal factor on $\Sigma$ which makes it positively curved.
Both statement are used 
to proof that $\TT^3$ does not admit a metric with positive scalar curvature;
the original proof is given in \cite{schoen-yau}.

Both statements admit straightforward generalization to higher dimensions
and they can be used to show non existence metric with positive scalar curvature on $\TT^m$ with $m\le 7$.
For $m=8$, the proof stops to work 
since in this dimension the area minimizing hypersurfaces might have singularities.
For example, 
any domain in the cone in $\RR^8$
defined by the identity
\[x^2_1+x^2_2+x^2_3+x^2_4=x^2_5+x^2_6+x^2_7+x^2_8\]
is area minimizing among the hypersurfaces with the same boundary.





%%%%%%%%%%%%%%%%%%%%%%%%%%%%%%%%%%%%%%%%%%%%%%%%%%
\parbf{No approximation.}
Fix an increasing function $\phi\:(0,r)\to \RR$
such that 
\[\phi''+(n-1)\cdot(\phi')^2+C=0.\]

Note that if $\Ric_{g_n}\ge C$, 
then the function 
$x\mapsto\phi(|p-x|_{g_n})$ is subharmonic.
In follows that, 
for arbitrary array of points $p_i$ 
and positive reals $\lambda_i$ the function $f_n\:M_n\to \RR$
defined by the formula
$$f(x)=\sum_i\lambda_i\cdot\phi(|p_i-x|_M)$$
is subharmonic.
In particular $f_n$ cannot admit a local minima in $M_n$.

Passing the limit as $n\to \infty$, we get that any function $f\:\mathbb{R}^m\to\mathbb{R}$
of the form 
$$f(x)=\sum_i\lambda_i\cdot\phi(|p_i-x|_{\ell_p})$$
does not admit a local minima in $\mathbb{R}^m$.

Arrive to a contradiction
by showing that if $p\ne 2$,
then there is an array
points $p_i$ and positive reals $\lambda_i$
such that the function 
$$f(x)=\sum_i\lambda_i\cdot\phi(|p_i-x|_{\ell_p})$$
has strict local minimum.\qeds

The argument given here is very close to the proof of Abresch--Gromoll inequality in \cite{abresch-gromoll}.
An alternative solution of this problem can be build on almost splitting theorem proved by  Jeff Cheeger and Tobias Colding in \cite{cheeger-colding}.





%%%%%%%%%%%%%%%%%%%%%%%%%%%%%%%%%%%%%%%%%%%%%%%%%%
\parbf{Area of spheres.}
Fix $r_0>0$.
Given $r>r_0$, choose a point $q$ on the distance $2\cdot r$ from $p$.

\begin{wrapfigure}{r}{30 mm}
\begin{lpic}[t(0 mm),b(0 mm),r(0 mm),l(0 mm)]{pics/calabi-yau(1)}
\lbl[t]{9.5,8.5;$p$}
\lbl[t]{28,2.5;$q$}
\lbl[b]{10,20.5;$\partial B(p,r)$}
\end{lpic}
\end{wrapfigure}

Note that any minimizing geodesic from $q$ to a point in $B(p,r_0)$
has to cross $\partial B(p,r)$.
By volume comparison, we get that 
\[\vol B(p,r_0)\le C_m\cdot r_0\cdot \area \partial B(p,r),\]
where $C_m$ is a constant depending only on the dimension $m=\dim M$,
say one can take $C_m\z=10^m$.\qeds


Applying the coarea formula, 
we get that volume growth of $M$ 
is at least linear and in particular it has infinite volume.
The latter was proved independently 
by Eugenio Calabi 
and Shing-Tung Yau \cite[see][]{calabi,yau-ricci}.

%%%%%%%%%%%%%%%%%%%%%%%%%%%%%%%%%%%%%%%%%%%%%%%%%%
\parbf{Curvature hollow.}
Construct a metric that the connected sum
$\RR^3\#\mathbb{S}^2\times\mathbb{S}^1$
which is flat outside a compact set and has non positive scalar curvature.
Further, note that such metric can be constructed in such a way that it has a closed geodesic $\gamma$ with trivial holonomy and with constant negative curvature in its a tubular neighborhood.

Let us cut tubular neighborhood $\mathbb{D}^2\times \mathbb{S}^1$ of $\gamma$ and glue in $\mathbb{S}^1\times \mathbb{D}^2$ with the swapped factors. 
Note that after this surgery we get $\RR^3$.

It remains to construct a metric $g$ on $\mathbb{S}^1\times \mathbb{D}^2$ with negative scalar curvature which 
is identical to the original metric on the neighborhood  of $\gamma$ near its boundary.
The needed metric $(\mathbb{S}^1\times \mathbb{D}^2,g)$ can be found among wrap products $\mathbb{S}^1\times_f \mathbb{D}^2$.
\qeds


This construction was given by Joachim Lohkamp in \cite{lohkamp},
he describes there yet an other equally simple construction.
In fact,
this  constructions produce 
$\mathbb{S}^1$-invariant hollows 
with negative Ricci curvature.

On the other hand there are no hollows with positive scalar curvature;
the latter is equivalent to the Positive Mass Conjecture.

%%%%%%%%%%%%%%%%%%%%%%%%%%%%%%%%%%%%%%%%%%%%%%%%%%
\parbf{Flat coordinate planes.}
Fix $\eps>0$ such that there is unique geodesic between any two points on distance $<\eps$ from the origin of $\RR^3$.

Consider three points $a$, $b$ and $c$ 
on the coordinate lines which are $\eps$-close 
to the origin.

Prove that the angles of the triangle $\triangle abc$
coincide with its model angles.
It follows that there is a flat geodesic triangle in $(\RR^3,g)$ with vertex at $a$, $b$ and $c$.

Use the family of constructed flat triangles 
to show that at any $x$ point in the $\tfrac\eps{10}$-neighborhood of the origin
the sectional curvature 
vanish in an open set of sectional directions.
The latter implies that the curvature is identically zero 
in this neighborhood.

Moving the origin and apply the same argument we get that the curvature is identically zero everywhere.
Hence the result follows. 
\qeds

This problem appears in the paper of Dmitri Panov and me \cite[see][]{panov-petrunin}; 
it is based on a lemma discovered by Sergei Buyalo in \cite{buyalo}.

%%%%%%%%%%%%%%%%%%%%%%%%%%%%%%%%%%%%%%%%%%%%%%%%%%
\parbf{Two-convexity.}
\textit{Morse-style solution.}
Equip $\RR^4$ with coordinates $(x,y,z,t)$.

Consider a generic linear function $\ell\:\RR^4\to\RR$
which is close to the sum of coordinates $x+y+z+t$.
Note that $\ell$
has non-degenerate critical points on $\partial K$ and all its critical values are different.

Consider the sets 
$$W_s=\set{w\in \RR^4\backslash K}{\ell(w)<s}.$$
Note that $W_{-1000}$ contains a closed curve, say $\alpha$, 
which is contactable in $\RR^4\backslash K$, 
but not constructable in the set $W_{-1000}$.

Set $s_0$ to be the infimum of the values $s$ such that
the $\alpha$ is contactable in $W_s$.

Note that $s_0$ is a critical value of $\ell$ on $\partial K$;
denote by $p_0$ the corresponding critical point.
By 2-convexity of $\RR^4\backslash K$,
the index of $p_0$ has to be at most $1$.
On the other hand, since the disc hangs at this point,
its index has to be at least $2$,
 a contradiction.\qeds

\parit{Alexandrov-style proof.}
Fix a constant metric $g$ on $\RR^4$.
According to the main result of Alexander Bishop and Berg in \cite{ABB}, $X_g=(\RR^4\backslash (\Int K),g)$ has non-positive curvature in the sense of Alexandrov.
In particular the universal cover of $\tilde X_g$ of $X_g$ is a $\CAT[0]$ space.

By rescaling $g$ and passing to the limit we obtain that universal Riemannian cover $Z_g$ of $(\RR^4,g)$ branching in the coordinate planes is a $\CAT[0]$ space.
Show that $Z_g$ is $\CAT[0]$ space if and only if the two planes are orthogonal with respect to $g$;
the latter leads to a contradiction.\qeds

The Morse-style proof is based on the idea described by Mikhael Gromov in \cite[][\S\textonehalf]{gromov-SGMC}.

Note that the closed $1$-neighborhood of these two planes has two-convex complement, but the boundary of this neighborhood is not smooth.