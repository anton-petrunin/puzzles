\csname @openrightfalse\endcsname
\chapter{Topology}

In this chapter we consider geometrical problems with strong topological flavor.
A typical introductory course in topology, say \cite{kosniowski},
contains all the necessary material.


%%%%%%%%%%%%%%%%%%%%%%%%%%%%%%%%%%%%%%%%%
\subsection*{Isotropy}\label{Isotropy}

Recall that an isotopy is a continuous one parameter family of embeddings.

\begin{pr}
Let $K_1$ and $K_2$ be homeomorphic closed subsets of the coordinate subspace $\RR^m$ in $\RR^{2\cdot m}$.
Show that there is a homeomorphism 
\[h\:\RR^{2\cdot m}\z\to \RR^{2\cdot m}\] 
such that $K_2=h(K_1)$.
Moreover, $h$ can be chosen to be isotopic to the identity map.
\end{pr}

%%%%%%%%%%%%%%%%%%%%%%%%%%%%%%%%%%%%%%%%%%%%%%%%%%
\parit{Semisolution.}
Choose a homeomorphism $\phi\:K_1\to K_2$.

By the Tietze extension theorem,
the homeomorphisms $\phi\:K_1\z\to K_2$ and $\phi^{-1}\:K_2\z\to K_1$ can be extended to continuous maps;
denote these maps by $f\:\RR^m\z\to \RR^m$ and $g\:\RR^m\z\to \RR^m$ respectively.

{

\begin{wrapfigure}{r}{61 mm}
\vskip-4mm
\centering
\includegraphics{mppics/pic-702}
\end{wrapfigure}

Consider homeomorphisms
$h_1$, $h_2$, and $h_3$ of $\RR^m\times\RR^m$
defined in the following way:
\begin{align*}
h_1(x,y)&=(x,y+f(x)),
\\
h_2(x,y)&=(x-g(y),y),
\\ 
h_3(x,y)&=(y,-x).
\end{align*}

}

It remains to observe that each homeomorphism $h_i$ is isotopic to the identity map and
\[K_2=h_3\circ h_2\circ h_1(K_1).\qedsin\]


This construction is due to Victor Klee \cite{klee} and it is called \emph{Klee's trick}.
This trick is used in the five-line proof of the Jordan separation theorem by Patrick Doyle \cite{doyle};
a proof of the separation theorem for embeddings $\mathbb{S}^n\hookrightarrow\mathbb{S}^{n+1}$
can be given using the same idea \cite{cohen}. 

The problem ``Monotonic homotopy'' on page \pageref{mono-homotopy} looks similar.

%%%%%%%%%%%%%%%%%%%%%%%%%%%%%%%%%%%%%%%%%
\subsection*{Immersed disks}\label{Immersed disks}

Two immersions $f_1$ and $f_2$ of the disk $\DD$ into the plane will be called {}\emph{essentially different} 
if there is no diffeomorphism $h\:\DD\z\to \DD$ such that
$f_1=f_2\circ h$.

\begin{pr} 
Construct two essentially different smooth immersions of the disk 
into the plane that coincide near the boundary. 
\end{pr}

%%%%%%%%%%%%%%%%%%%%%%%%%%%%%%%%%%%%%%%%%
\subsection*{Positive Dehn twist}\label{Positive Dehn twist} 

\begin{wrapfigure}{r}{43 mm}
\begin{lpic}[t(-4 mm),b(0 mm),r(0 mm),l(0 mm)]{asy/dehn-twist()}
\lbl[b]{20,10;$\xrightarrow{\ h\ }$}
\end{lpic} 
\end{wrapfigure}

Let $\Sigma$ be a surface and 
\[\gamma\:\RR/\ZZ\z\to\Sigma\] 
be a non-contractible closed simple curve.
Let $U_\gamma$ be a neighborhood of $\gamma$ that admits a parametrization 
\[\iota\:\RR/\ZZ\times (0,1)\to U_\gamma.\]
A \index{Dehn twist}\emph{Dehn twist} along $\gamma$ is a homeomorphism $h\:\Sigma\z\to\Sigma$
that is the identity outside of $U_\gamma$ and 
such that
\[\iota^{-1}\circ h\circ \iota\:(x,y)\mapsto(x+y,y).\]

If $\Sigma$ is oriented 
and $\iota$ is orientation preserving,
then the Dehn twist described above is called {}\emph{positive}.

\begin{pr}
Let $\Sigma$ be a compact oriented surface with nonempty boundary.
Prove that any composition of positive Dehn twists of $\Sigma$ is not homotopic to the identity relative to the boundary.

In other words, any product of positive Dehn twists represents a nontrivial class in the mapping class group of $\Sigma$.
\end{pr}


%%%%%%%%%%%%%%%%%%%%%%%%%%%%%%%%%%%%%%%%%
\subsection*{Conic neighborhood}
\label{Conic neighborhood}

Let $p$ be a point in a topological space $X$.
We say that an open neighborhood $U\ni p$ is \index{conic neighborhood}\emph{conic}
if there is a homeomorphism from a cone
to $U$ that sends the vertex to $p$.

\begin{pr}  
Show that any two conic neighborhoods of one point are homeomorphic to each other.
\end{pr}

Note that two cones $\mathop{\rm Cone}(\Sigma_1)$ and $\mathop{\rm Cone}(\Sigma_2)$ might be homeomorphic while $\Sigma_1$ and $\Sigma_2$ are not;
existence of such examples follow from the double suspension theorem.

%%%%%%%%%%%%%%%%%%%%%%%%%%%%%%%%%%%%%%%%%
\subsection*{Unknots\easy}\label{No knots}

\begin{pr}
Prove that the set of smooth embeddings $f\:\mathbb{S}^1\z\to\RR^3$ equipped with the $C^0$-topology 
forms a connected space.
\end{pr}

%%%%%%%%%%%%%%%%%%%%%%%%%%%%%%%%%%%%%%%%%
\subsection*{Stabilization}\label{Simple stabilization}

\begin{pr}
Construct two compact subsets $K_1, K_2\subset\RR^2$ such that
$K_1$ is not homeomorphic to $K_2$, but $K_1\times[0,1]$ is homeomorphic to $K_2\z\times[0,1]$.
\end{pr}

%%%%%%%%%%%%%%%%%%%%%%%%%%%%%%%%%%%%%%%%%
\subsection*{Homeomorphism of a cube}\label{Homeomorphism of cube}

\begin{pr}
Let $\square$ be a cube in $\RR^m$
and $h\:\square\z\to\square$ be
a homeomorphism that sends each face of $\square$ to itself.
Extend $h$ to a homeomorphism $f\:\RR^m\z\to\RR^m$ that coincides with the identity map outside of a bounded set.    
\end{pr}

%%%%%%%%%%%%%%%%%%%%%%%%%%%%%%%%%%%%%%%%%
\subsection*{Finite topological space\easy}\label{Finite topological space}


\begin{pr}
Given a finite topological space $F$, 
construct a finite simplicial complex $K$
that admits a weak homotopy equivalence  $K\to F$. 
\end{pr}

%%%%%%%%%%%%%%%%%%%%%%%%%%%%%%%%%%%%%%%%%
\subsection*{Dense homeomorphism\easy}\label{Dense homeomorphism}

\begin{pr}
Denote by $\mathcal{H}$ be the set of all orientation preserving homeomorphisms $\mathbb {S}^2\z\to\mathbb {S}^2$ 
equipped with the $C^0$-metric.
Show that there is a homeomorphism $h\in \mathcal{H}$ such that its conjugations $a\circ h\circ a^{-1}$ for all $a\in\mathcal{H}$ form a dense set in $\mathcal{H}$.
 
\end{pr}

%%%%%%%%%%%%%%%%%%%%%%%%%%%%%%%%%%%%%%%%%
\subsection*{Simple path\easy}
\label{Simple path}

\begin{pr}
Let $p$ and $q$ be distinct points in a Hausdorff topological space $X$.
Assume that $p$ and $q$ are connected by a path.
Show that they can be connected by a simple path;
that is, there is an injective continuous map $\beta\:[0,1]\z\to X$
such that $\beta(0)=p$ and $\beta(1)=q$.
\end{pr}

(This statement might be intuitively obvious, but its proof is not simple.)

\subsection*{Path on a surface\easy}
\label{Path on a surface}

\begin{pr}
Show that any path with distinct ends in a surface is homotopic (relative to the ends) to a simple path.  
\end{pr}






{

\begin{wrapfigure}{r}{29 mm}
\vskip-4mm
\centering
\includegraphics{mppics/pic-704}
\bigskip
\includegraphics{mppics/pic-706}
\end{wrapfigure}

\section*{Semisolutions}

%%%%%%%%%%%%%%%%%%%%%%%%%%%%%%%%%%%%%%%%%%%%%%%%%%
\parbf{Immersed disks.}
Both circles on the picture bound essentially different disks.



\medskip

On the first diagram, the dashed lines and the solid lines together bound three embedded disks;
gluing these disks along the dashed lines gives the first immersion.
The reflection of this immersion across the vertical line of symmetry gives another immersion which is essentially different.
\qeds

}


It is a good exercise to count the essentially different disks in the second example. 
(The answer is 5.) 

{

\begin{wrapfigure}{r}{30 mm}
\vskip-7mm
\centering
\includegraphics{mppics/pic-708}
\end{wrapfigure}

The existence of examples of that type is generally attributed to John Milnor \cite{bennequin}.

An easier problem would be to construct two essentially different immersions of annuli with the same boundary curves; a solution is shown on the picture \cite[for more details and references see][]{eppstein-mumford}.

}

%%%%%%%%%%%%%%%%%%%%%%%%%%%%%%%%%%%%%%%%%%%%%%%%%%
\parbf{Positive Dehn twist.}
Consider the universal covering 
$f\:\tilde\Sigma\z\to\Sigma$.
The surface $\tilde \Sigma$ has a boundary and it comes with the orientation induced from $\Sigma$.



Choose a point $x_0$ on the boundary $\partial \tilde \Sigma$.
Given two other points $y$ and $z$ in $\partial \tilde \Sigma$, we will write
$z\succ y$ if $y$ lies on the right side from some simple curve from $x_0$ to $z$ in $\tilde\Sigma$.
Note that  $\succ $ defines a linear order on $\partial\tilde\Sigma\backslash\{x_0\}$.
We will write $z \succeq y$ 
if $z\succ y$ or $z=y$.

{

\begin{wrapfigure}{r}{27 mm}
\vskip-0mm
\centering
\includegraphics{mppics/pic-710}
\end{wrapfigure}

Note that any homeomorphism $h\:\Sigma\to\Sigma$ identical on the boundary
lifts to the unique homeomorphism $\tilde h\:\tilde \Sigma\to\tilde\Sigma$ 
such that $\tilde h(x_0)\z=x_0$.
The following claim is the key step in the proof:

}

\begin{cl}{$({*})$} 
If $h$ is a positive Dehn twist along a closed curve $\gamma$,
then $y\succeq \tilde h(y)$ for any $y\in\partial\tilde\Sigma\backslash\{x_0\}$
and $y_0\succ\tilde h(y_0)$ for some $y_0\in\partial\tilde\Sigma\backslash\{x_0\}$.
\end{cl}

Note that the problem follows from~$({*})$.
Indeed, the property in $({*})$ is a homotopy invariant 
and it survives under compositions of maps.

\medskip

If $\Sigma$ is not an annulus,
then by the uniformization theorem we can assume that $\Sigma$ has a  hyperbolic metric with geodesic boundary; 
the lifted metric on $\tilde\Sigma$ has the same properties.
Furthermore, we can assume that (1) $\gamma$ is a closed geodesic,
(2) the parametrization $\iota\:\RR/\ZZ\z\times (0,1)\to U_\gamma$ from the definition of Dehn twist is rotationally symmetric 
and (3) for any $u\in \RR/\ZZ$ the arc $\iota(u\times (0,1))$ is a geodesic perpendicular to~$\gamma$. 

Consider the polar coordinates $(\phi,\rho)$ on $\tilde\Sigma$ with the origin at~$x_0$;
since $x_0$ lies on the boundary, the angle coordinate $\phi$ is defined in $[0,\pi]$. 
By construction of Dehn twist, we get 
\[\phi(x)\ge \phi\circ\tilde h(x)\]
for any $x\ne x_0$ 
and if the geodesic $[x_0x]$ crosses $f^{-1}(U_\gamma)$, then 
\[\phi(x)> \phi\circ\tilde h(x).\]
In particular, if $x$ lies on the boundary then $\tilde h(x)$ lies on the right of the geodesic $[x_0x]$; hence the claim $({*})$ follows. 

\begin{figure}[!ht]
\vskip0mm
\centering
\includegraphics{mppics/pic-712}
\end{figure}

If $\Sigma$ is an annulus, then the same argument works except we have to choose a flat metric on $\Sigma$.
In this case $\tilde \Sigma$ is a strip between two parallel lines in the plane, see the diagram.
\qeds

The problem was suggested by Rostislav Matveyev.
It is instructive to solve the following problem.

\begin{pr}
Construct a composition of positive Dehn twists on a compact oriented surface without boundary that is homotopic to the identity. 
\end{pr}




%%%%%%%%%%%%%%%%%%%%%%%%%%%%%%%%%%%%%%%%%%%%%%%%%%
\parbf{Conic neighborhood.}
Let $V$ and $W$ be two conic neighborhoods of~$p$.
Without loss of generality, we may assume that $V\Subset W$;
that is, the closure of $V$ lies in $W$.

We will need to construct a sequence of embeddings $f_n\:V\to W$
such that 
\begin{itemize}
\item 
For any compact set $K\subset V$ 
there is a positive integer $n=n_K$ such that 
$f_n(k)=f_m(k)$ for any $k\in K$ and $m, n \ge n_K$.
\item For any point $w\in W$ there is a point $v\in V$ such that $f_n(v)=w$ for all large $n$.
\end{itemize}

Note that once such a sequence is constructed, $f\:V\to W$ defined by $f(v)=f_n(v)$ for all large values of $n$ gives the needed homeomorphism.

The sequence $f_n$ can be constructed recursively
\[f_{n+1}=\Psi_n\circ f_n\circ \Phi_n,\]
where $\Phi_n\:V\to V$ 
and $\Psi_n\:W\to W$ 
are homeomorphisms
of the form 
\[\Phi_n(x)=\phi_n(x)\ast x\quad \text{and}\quad \Phi_n(x)=\psi_n(x)\star x,\]
where $\phi_n\:V\to \RR_{\ge 0}$, $\psi_n\:W\to \RR_{\ge 0}$ are suitable continuous functions;
``$\ast$'' and ``$\star$'' denote the {}\emph{multiplication} in the cone structures of $V$ and $W$ respectively.\qeds


The problem is due to Kyung Whan Kwun \cite{kwun}.

%%%%%%%%%%%%%%%%%%%%%%%%%%%%%%%%%%%%%%%%%%%%%%%%%%
\parbf{Unknots.}

\begin{figure}[!ht]
\vskip0mm
\centering
\includegraphics{mppics/pic-714}
\end{figure}

Observe that it is possible to draw an arbitrary tight knot 
while keeping it smoothly embedded at all times including the last moment.\qeds


\begin{wrapfigure}[6]{r}{50 mm}
\vskip-3mm
\centering
\includegraphics{mppics/pic-716}
\end{wrapfigure}

This problem was suggested by Greg Kuperberg.

%%%%%%%%%%%%%%%%%%%%%%%%%%%%%%%%%%%%%%%%%%%%%%%%%%

\parbf{Stabilization.}
The example can be guessed from the diagram.

The two sets $K_1$ and $K_2$ are subspaces of the plane, 
each one being a closed annulus with two attached line segments.
In $K_1$ one segment is attached from inside and the other from outside and 
in $K_2$ both segments are attached from outside.

The product spaces $K_1\times[0,1]$ and $K_2\times[0,1]$ are solid tori with attached rectangles.
A homeomorphism $K_1\times[0,1]\z\to K_2\times[0,1]$ can be constructed by twisting a part of one solid torus.

To prove the nonexistence of a homeomorphism $K_1\to K_2$ consider the sets of cut points $V_i\subset K_i$ and the sets $W_i\subset K_i$ of points that admit a punctured simply connected neighborhood.
Note that the set $V_i$ is the union of the attached line segments 
and $W_i$ is the boundary of the annulus without points where the segments are attached.
Note that $V_i\cup W_i=\partial K_i$;
in particular, a homeomorphism $K_1\to K_2$ (if it exists) sends $\partial K_1$ to $\partial K_2$.

Finally note that each $\partial K_i$ has two connected components and 
$V_1$ intersects both components of $\partial K_1$
while $V_2$ lies in one component of $\partial K_2$.
Hence $K_1\ncong K_2$.
\qeds

It should be a very old puzzle;
I learned it from Maria Goluzina around 1988.


%%%%%%%%%%%%%%%%%%%%%%%%%%%%%%%%%%%%%%%%%%%%%%%%%%
\parbf{Homeomorphism of a cube.}
Let us extend the homeomorphism $h$ to $\RR^m$ by reflecting the cube across its facets.
We get a homeomorphism $\tilde h\:\RR^m\to\RR^m$ such that $\tilde h(x)=h(x)$ for any $x\in\square$ and 
\[\tilde h\circ\gamma=\gamma\circ \tilde h,\]
where $\gamma$ is any reflection with respect to the facets of the cube.

Without loss of generality we may assume that the cube $\square$ is inscribed in the unit sphere centered at the origin of $\RR^m$.
In this case $\tilde h$ has \index{displacement}\emph{displacement} at most $2$;
that is, 
\[|\tilde h(x)-x|\le 2\]
for any $x\in\RR^m$.

Choose a smooth increasing function $\phi\:\RR_{\ge 0} \to\RR$ such that
$\phi(r)\z=r$
for $r\le 1$ and $\phi(r)\to 2$ as $r\to\infty$.

{

\begin{wrapfigure}{o}{41 mm}
\vskip-0mm
\centering
\includegraphics{mppics/pic-717}
\vskip0mm
\end{wrapfigure}

Equip $\RR^m$ with polar coordinates $(u,r)$, 
where $u\in\mathbb{S}^{m-1}$, $r\z\ge 0$.
Consider a homeomorphism $\Phi$ from $\RR^m$ to an open ball of radius $2$
defined by 
\[\Phi(u,r)\z=(u,\phi(r)).\]
Set 
\[
f(x)=\left[
\begin{aligned}
&x&&\text{if}\ |x|\ge 2,
\\
&\Phi\circ \tilde h \circ \Phi^{-1}(x)&&\text{if}\ |x|< 2.
\end{aligned}
\right.
\]

}

It remains to observe that $f\:\RR^m\to\RR^m$ is a solution.
\qeds

This problem is stripped from a proof of Robion Kirby \cite{kirby}.
The condition that each face is mapped to itself can be removed and 
instead of homeomorphism one could take any embedding close to the identity.

An interesting twist to this idea was given by Dennis Sullivan \cite{sullivan}.
Instead of the discrete group of motions of the Euclidean space,
he uses a discrete group of motions of the hyperbolic space in the conformal disk model.
To see the idea, note that the construction of $\tilde h$ can be done for a Coxeter polytope in the hyperbolic space instead of a cube.%
\footnote{By Vinberg's theorem \cite{vinberg, vinberg-strong} hyperbolic space of large dimension has no Coxeter polytopes, but the idea works anyway after some modifications.}
Then the constructed map $\tilde h$
coincides with the identity on the absolute and therefore the last ``shrinking'' step in the proof above is not needed.
Moreover, 
if the original homeomorphism is bi-Lipschitz,
then the Sullivan construction produces a bi-Lipschitz homeomorphism ---
this is its main advantage.

  

%%%%%%%%%%%%%%%%%%%%%%%%%%%%%%%%%%%%%%%%%%%%%%%%%%
\parbf{Finite topological space.}
Given a point $p\in F$,
denote by $O_p$ the minimal open set in $F$ containing $p$. 
Note that we can assume that $F$ is a connected $T_0$-space;
in particular, $O_p=O_q$ if and only if $p=q$.

Let us write $p\preccurlyeq q$ 
if $O_p\subset O_q$.
The relation $\preccurlyeq$ is a partial order on~$F$.

Let us construct a simplicial complex $K$ 
by taking $F$ as the set of vertices
and declaring a collection of vertices to be a simplex 
if it can be linearly ordered with respect to $\preccurlyeq$.

Given $k\in K$,
consider the minimal simplex $(f_0,\dots,f_m)\ni k$;
we can assume that $f_0\preccurlyeq \dots\preccurlyeq f_m$.
Set $h\:k\mapsto f_0$;
it defines a map $K\to F$.

It remains to check that $h$ is continuous 
and induces isomorphisms for all the homotopy groups.
\qeds

In a similar fashion, one can construct a finite topological space $F$ for any given simplicial complex $K$ 
such that 
there is a weak homotopy equivalence $K\to F$.
Both constructions are due to Pavel Alexandrov
\cite{alexandrov-finite,mccord}.

\parbf{Dense homeomorphism.}
Note that there is a countable set of homeomorphisms $h_1,h_2,\dots$ that is dense in $\mathcal{H}$
such that
each $h_n$ fixes all the points outside an open round disk, say $D_n$.

Choose a countable disjoint collection of round disks $D_n'$.
Consider the homeomorphism $h\:\mathbb S^2\to \mathbb S^2$
that fixes all the points outside of $\bigcup_nD'_n$ and
for each $n$,
the restriction $h|_{D_n'}$ is conjugate to $h_n|_{D_n}$. 


Note that for large $n$, the homeomorphism $h$ is conjugate to a homeomorphism close to $h_n$.
Therefore $h$ is a solution.
\qeds

The problem was mentioned by Frederic Le Rox \cite{rox} on a problem section at a conference in Oberwolfach, 
where he also conjectured that this is not true for the area-preserving homeomorphisms.
An affirmative answer to this conjecture was given by Daniel Dore, Andrew Hanlon and Sobhan Seyfaddini 
\cite{dore-hanlon,seyfaddini}.
In particular it implies the following seemingly evident but nontrivial statement.

\begin{pr}
Given $\eps>0$, there is $\delta>0$ such that 
\[\Omega\cap h(\Omega)\ne\emptyset\]
for any topological disk $\Omega\subset \mathbb{S}^2$ with area at least $\eps$
and 
any area-preserving homeomorphism $h\:\mathbb{S}^2\to\mathbb{S}^2$ with displacement at most $\delta$;
that is, such that $|h(x)-x|_{\mathbb{S}^2}<\delta$ for any $x\in \mathbb{S}^2$. 
\end{pr}


%%%%%%%%%%%%%%%%%%%%%%%%%%%%%%%%%%%%%%%%%%%%%%%%%%
\parbf{Simple path.}
We will give two solutions, the first one is elementary and the second one is involved. 

\parit{First solution.}
Let $\alpha$ be a path connecting $p$ to $q$.

Passing to a subinterval if necessary,
we can assume that $\alpha(t)\ne p,q$ for $t\ne0,1$.

An open set $\Omega$ in $(0,1)$ will be called {}\emph{suitable}
if for any connected component $(a,b)$ of $\Omega$ we have $\alpha(a)=\alpha(b)$.
Since the union of nested suitable sets is suitable, we can find a maximal suitable set $\hat \Omega$.

Define $\beta(t)=\alpha(a)$ for any $t$ in a connected component $(a,b)\subset\Omega$.
Note that $\beta$ is continuous and monotonic;
that is, for any $x\in [0,1]$ the set $\beta^{-1}\{\beta(x)\}$ is connected.

It remains to reparametrize $\beta$ to make it injective.
In other words we need to construct a non-decreasing surjective function $\tau\:[0,1]\z\to[0,1]$ such that 
$\tau(t_1)=\tau(t_2)$ if and only if there is a connected component $(a,b)$ such that $t_1,t_2\z\in [a,b]$.
The construction is similar to the construction of the devil's staircase.
\qeds

\parit{Second solution.}
Note that one can assume that $X$ coincides with the image of $\alpha$.
In particular, $X$ is a connected locally connected compact Hausdorff space.

Any such space admits a length-metric.
This statement is not at all trivial;
it was conjectured by Karl Menger \cite{menger}
and proved independently 
by R.~H.~Bing  \cite{bing-length-0, bing-length-1} 
and Edwin Moise \cite{moise}.

It remains to consider a geodesic path from $p$ to $q$.
\qeds

The problem was inspired by a lemma 
proved by 
Alexander Lytchak
and Stefan Wenger \cite[see 7.13 in][]{lytchak-wenger}.


%%%%%%%%%%%%%%%%%%%%%%%%%%%%%%%%%%%%%%%%%%%%%%%%%%
\parbf{Path on a surface.}
Denote the surface by $\Sigma$, assume that the path runs from $p$ to $q$.
The following picture suggests an idea for an induction proof on the number of self-crossings.

\begin{figure}[!ht]
\vskip0mm
\centering
\includegraphics{mppics/pic-718}
\end{figure}

To do the proof formally,
let us present the path as a concatenation $\alpha*\beta$ of two paths  such that $\alpha$ is simple
and $\beta$ does not pass thru~$p$.
We can assume that $\beta\:[0,1]\to \Sigma$ is smooth.

Choose a smooth time dependent vector field $V_t$ on $\Sigma$ such that
\[V_t(\beta(t))=\beta'(t)\quad\text{and}\quad V_t(p)=0\]
for any $t\in[0,1]$. 

Consider the flow $\Phi^t\:\Sigma\to \Sigma$ along $V_t$;
that is,
\[\Phi^0(x)=x\quad\text{and}\quad \tfrac{d}{dt}(\Phi^t(x))=V_t(\Phi^t(x))\]
for any $t\in[0,1]$ and $x\in \Sigma$.
The map $\Phi^1\:\Sigma\to \Sigma$ is a diffeomorphism;
in particular $\Phi^1$ sends the simple path $\alpha$ to a simple path $\alpha_1=\Phi^1\circ\alpha$.
By construction $\alpha_1(1)=q$. 
Since $V_t(p)=0$ for any $t$, we have $\alpha_1(0)\z=p$.
That is, the path $\alpha_1$ runs from $p$ to $q$.

It remains to show that $\alpha_1$ is homotopic to $\alpha*\beta$ relative to the ends.
Set $\alpha_\tau=\Phi^\tau\circ\alpha$ and denote by  $\beta_\tau$ the path running along $\beta$ from $\beta(\tau)$ to $q$;
that is, 
\[\beta_\tau(t)=\beta(\tau+\tfrac1{1-\tau}\cdot t).\]
The concatenation $\alpha_\tau*\beta_\tau$ provides a homotopy from $\alpha*\beta$ to $\alpha_1*\beta_1$. 
Since $\beta_1$ is a constant path, $\alpha*\beta$ is homotopic to $\alpha_1$.
Hence the statement follows.
\qeds

This is a stripped version of the problem suggested by Jaros{\l}aw K\k{e}dra \cite{One-step}; 
it was used by Michael Khanevsky \cite[Lemma 3 in][]{khanevsky}.
